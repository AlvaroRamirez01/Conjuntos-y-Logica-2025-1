\begin{solution}
\textbf{P.D:} que \( S \) es un orden para \( A/R \):

La relación \( S \) en \( A/R \) está definida de la siguiente manera:

\[
[a]_R S [b]_R \iff \exists x \in [a]_R, \, y \in [b]_R \text{ tales que } x \rightarrow y \text{ es una tautología}.
\]

Para demostrar que \( S \) es un orden, debemos verificar que cumple las tres propiedades que definen un \textbf{orden parcial}: \textbf{reflexividad}, \textbf{antisimetría} y \textbf{transitividad}.

\begin{enumerate}[label=\arabic*.]
\item \textbf{Reflexividad:}

Queremos probar que \( [a]_R S [a]_R \) para cualquier \( [a]_R \in A/R \).

Por definición de \( S \), esto significa que existe \( x \in [a]_R \) tal que \( x \rightarrow x \) es una tautología. Esto es cierto, ya que \( x \rightarrow x \) es siempre una tautología, independientemente de \( x \). Por lo tanto, \( [a]_R S [a]_R \), y la relación \( S \) es reflexiva.

\item \textbf{Antisimetría:}

Debemos probar que si \( [a]_R S [b]_R \) y \( [b]_R S [a]_R \), entonces \( [a]_R = [b]_R \).

Por definición de \( S \), si \( [a]_R S [b]_R \), existen \( x \in [a]_R \) y \( y \in [b]_R \) tales que \( x \rightarrow y \) es una tautología. De igual forma, si \( [b]_R S [a]_R \), existen \( y \in [b]_R \) y \( x \in [a]_R \) tales que \( y \rightarrow x \) es una tautología. 

Ahora, si \( x \rightarrow y \) y \( y \rightarrow x \) son ambas tautologías, por definición de equivalencia lógica, esto implica que \( x \leftrightarrow y \) es una tautología. Como \( x \in [a]_R \) y \( y \in [b]_R \), esto significa que \( [a]_R = [b]_R \). Por lo tanto, la relación \( S \) es antisimétrica.

\item \textbf{Transitividad:}

Debemos probar que si \( [a]_R S [b]_R \) y \( [b]_R S [c]_R \), entonces \( [a]_R S [c]_R \).

Por definición, si \( [a]_R S [b]_R \), existen \( x \in [a]_R \) y \( y \in [b]_R \) tales que \( x \rightarrow y \) es una tautología. Además, si \( [b]_R S [c]_R \), existen \( y' \in [b]_R \) y \( z \in [c]_R \) tales que \( y' \rightarrow z \) es una tautología. Dado que \( y \in [b]_R \) y \( y' \in [b]_R \), tenemos que \( y \leftrightarrow y' \) es una tautología, lo que implica que \( x \rightarrow z \) es una tautología.

Por lo tanto, \( [a]_R S [c]_R \), y la relación \( S \) es transitiva.
\end{enumerate}

\textbf{Conclusión:}

Dado que la relación \( S \) es reflexiva, antisimétrica y transitiva, podemos concluir que \( S \) es un \textbf{orden parcial} sobre \( A/R \).

\textbf{Elemento mayor y menor en el orden:}

En un orden parcial, el \textbf{elemento mayor} es aquel que está relacionado con todos los demás elementos, y el \textbf{elemento menor} es aquel que está relacionado por todos los demás elementos. En este caso, necesitamos identificar los elementos correspondientes en términos de las tautologías.

\begin{itemize}
    \item El \textbf{elemento mayor} es la clase de equivalencia de la \textbf{tautología}. Esto se debe a que para cualquier fórmula \( a \), \( \top \rightarrow a \) es siempre una tautología. Por lo tanto, la clase de equivalencia \( [\top]_R \) (donde \( \top \) es una tautología) es el mayor elemento.
  
    \item El \textbf{elemento menor} es la clase de equivalencia de la \textbf{contradicción}. Para cualquier fórmula \( a \), \( a \rightarrow \bot \) (donde \( \bot \) es una contradicción) es siempre una tautología. Esto significa que la clase de equivalencia \( [\bot]_R \) (donde \( \bot \) es una contradicción) es el menor elemento.
\end{itemize}

Por lo tanto, el orden definido por \( S \) tiene un elemento mayor \( [\top]_R \) y un elemento menor \( [\bot]_R \).
\end{solution}