\begin{solution}
\textbf{P.D:} que \( R \) es una relación de equivalencia:

La relación \( R \) en \( \mathbb{R}^2 \) está definida de la siguiente manera:
\[
(a, b) R (c, d) \iff (a - c) \in \mathbb{Z} \text{ y } (b - d) \in \mathbb{Z}
\]
Para demostrar que \( R \) es una relación de equivalencia, debemos verificar que cumple las tres propiedades de una relación de equivalencia: \textbf{reflexiva}, \textbf{simétrica} y \textbf{transitiva}.

\begin{enumerate}[label=\arabic*.]
    \item \textbf{Reflexividad}:
    Debemos probar que \( (a, b) R (a, b) \) para cualquier \( (a, b) \in \mathbb{R}^2 \).
    
    \[
    a - a = 0 \in \mathbb{Z} \quad \text{y} \quad b - b = 0 \in \mathbb{Z}
    \]
    Por lo tanto, \( (a, b) R (a, b) \), lo que demuestra que la relación es reflexiva.
    \item \textbf{Simetría}:
    Debemos probar que si \( (a, b) R (c, d) \), entonces \( (c, d) R (a, b) \).
    
    Si \( (a, b) R (c, d) \), entonces \( a - c \in \mathbb{Z} \) y \( b - d \in \mathbb{Z} \). Como \( \mathbb{Z} \) es cerrado bajo la negación, tenemos que:
    \[
    c - a = -(a - c) \in \mathbb{Z} \quad \text{y} \quad d - b = -(b - d) \in \mathbb{Z}
    \]
    Por lo tanto, \( (c, d) R (a, b) \), lo que demuestra que la relación es simétrica.
    
    \item \textbf{Transitividad}:
    Debemos probar que si \( (a, b) R (c, d) \) y \( (c, d) R (e, f) \), entonces \( (a, b) R (e, f) \).
    
    Si \( (a, b) R (c, d) \), entonces \( a - c \in \mathbb{Z} \) y \( b - d \in \mathbb{Z} \). Si \( (c, d) R (e, f) \), entonces \( c - e \in \mathbb{Z} \) y \( d - f \in \mathbb{Z} \). Sumando estas ecuaciones, obtenemos:
    \[
    (a - c) + (c - e) = a - e \in \mathbb{Z} \quad \text{y} \quad (b - d) + (d - f) = b - f \in \mathbb{Z}
    \]
    Por lo tanto, \( (a, b) R (e, f) \), lo que demuestra que la relación es transitiva.
\end{enumerate}

\textbf{Conclusión:}
Como la relación \( R \) es reflexiva, simétrica y transitiva, podemos concluir que es una \textbf{relación de equivalencia} en \( \mathbb{R}^2 \).

\textbf{Conjunto cociente:}

El conjunto cociente bajo la relación \( R \) consiste en las clases de equivalencia de los puntos \( (a, b) \in \mathbb{R}^2 \). Dos puntos \( (a, b) \) y \( (c, d) \) están en la misma clase de equivalencia si:
\[
a - c \in \mathbb{Z} \quad \text{y} \quad b - d \in \mathbb{Z}
\]
Esto implica que los puntos \( (a, b) \) y \( (c, d) \) están en la misma clase de equivalencia cada clase de equivalencia corresponde a una traslación entera en ambas coordenadas.

El conjunto cociente \( \mathbb{R}^2 / R \) es isomorfo al \textbf{torno de lados 1}, que es un conjunto que representa los puntos de \( \mathbb{R}^2 \) módulo las traslaciones enteras:
\[
[0, 1) \times [0, 1)
\]
donde cada punto en \( [0, 1) \times [0, 1) \) representa una clase de equivalencia de puntos en \( \mathbb{R}^2 \) bajo la relación \( R \).
\end{solution}