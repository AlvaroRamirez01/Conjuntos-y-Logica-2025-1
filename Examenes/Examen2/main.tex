\documentclass[11pt,answers]{exam}
\usepackage[spanish]{babel}
\usepackage[utf8]{inputenc}
\usepackage[T1]{fontenc}
\usepackage{amsmath,amssymb,amsfonts}
\usepackage{graphicx}
\usepackage{colortbl}
\usepackage{xcolor}
\usepackage{multirow}
\usepackage{float}
\usepackage{enumitem}
\usepackage{algorithm}
\usepackage{mathrsfs}
\usepackage{array} % Para controlar el ancho de las celdas
\usepackage{enumitem} % Para personalizar listas
\usepackage{listings}% http://ctan.org/pkg/listings
\usepackage{hyperref} % for hyperlinks
\usepackage{amsmath} % para matemáticas mejoradas
\usepackage{algorithm} % para escribir pseudocódigos
\usepackage{algpseudocode} % para escribir pseudocódigos
\usepackage[version=4]{mhchem}
\usepackage{stmaryrd}

% \usepackage{etoolbox}

% \AtBeginEnvironment{align}{\setcounter{equation}{0}}

\renewcommand{\solutiontitle}{\noindent\textbf{Solución:}\par\noindent}

\lstset{
  basicstyle=\ttfamily,
  mathescape
}

\graphicspath{{public/}}

\setlength{\topmargin}{-.5in} \setlength{\textheight}{9.25in}
\setlength{\oddsidemargin}{-0.5in} \setlength{\textwidth}{7.2in}


\begin{document}
\begin{center}
    \newcommand{\HRule}{\rule{\linewidth}{0.5mm}}
    \begin{minipage}{0.48\textwidth} 
        \begin{flushleft}
            \includegraphics[scale = 0.08]{../../../public/logo_unam.png}
        \end{flushleft}
    \end{minipage}
    \begin{minipage}{0.48\textwidth} 
        \begin{flushright}
            \includegraphics[scale =0.22]{../../../public/logo_ciencias.png}
        \end{flushright}
    \end{minipage}
    \vspace*{-1.5cm}						
    \textsc{\huge Nacional Autónoma de México \\ \vspace{-4px} Universidad }\\[2cm]	
    \textsc{\LARGE Facultad de Ciencias}\\[1.5cm]
    \vspace*{1cm}					
        \HRule \\[0.7cm]							
            { \huge \bfseries Tarea 03}\\[0.4cm]	
        \HRule \\[1.5cm]						    
    \begin{minipage}{0.52\textwidth}													
        \begin{flushleft} \large	
            \small
            \vspace{-0.6cm}	
            \vspace{-0.6cm}	
                \emph{Alumno:}\\
               Ramírez López Alvaro. 316276355\\
            \vspace*{2cm}
        \end{flushleft}																		
        \end{minipage}		
    \begin{minipage}{0.46\textwidth}		
        \vspace{-0.6cm}											
        \begin{flushright} \large						
            \small										
            \emph{Profesor:} Jesús Villagómez Chávez	\\
            \emph{Ayudantes:}
                Gabriela Peña Franco	 \\
                Martha Rubí Gutiérrez González	 \\
        \end{flushright}																
    \end{minipage}	
    \vspace*{1cm}
    \vspace{2cm}
    \begin{center}						
        {\large 9 de septiembre de 2024}
    \end{center}  						
\end{center}	
\textbf{}
\newpage

Expón detalladamente tus respuestas:

\begin{enumerate}
    \item Sean \( \mathscr{U} \) universo y \( \mathscr{F} \) una familia no vacía de subconjuntos de \( \mathscr{U} \). ¿Es cierto que 
    \[
    \mathscr{P} \left( \bigcap_{X \in \mathscr{F}} X \right) = \bigcap_{X \in \mathscr{F}} \mathscr{P}(X)?
    \]
    ¿Y que 
    \[
    \mathscr{P} \left( \bigcup_{X \in \mathscr{F}} X \right) = \bigcup_{X \in \mathscr{F}} \mathscr{P}(X)?
    \]

    \begin{solution}
\begin{enumerate}
    \item \textbf{Primera proposición:}
   \[
   \mathscr{P} \left( \bigcap_{X \in \mathscr{F}} X \right) = \bigcap_{X \in \mathscr{F}} \mathscr{P}(X)?
   \]
   \textbf{Interpretación:}

   \( \mathscr{P}(A) \) es el conjunto potencia de \( A \), es decir, el conjunto de todos los subconjuntos de \( A \). La proposición está afirmando que el conjunto potencia de la intersección de todos los conjuntos en \( \mathscr{F} \) es igual a la intersección de los conjuntos potencia de los conjuntos en \( \mathscr{F} \).

   \textbf{Análisis:}

   Esto \textbf{seria cierto}. La razón es que el conjunto potencia de la intersección de una familia de conjuntos \( \mathscr{F} \) contiene los subconjuntos de la intersección, pero la intersección de los conjuntos potencia contiene los subconjuntos de cada conjunto individual en \( \mathscr{F} \), no necesariamente aquellos que están en la intersección. Vamos a sustentarlo con un ejemplo:

   \begin{itemize}
        \item Sean \( A = \{1, 2\} \) y \( B = \{2, 3\} \).
        \item Entonces, \( A \cap B = \{2\} \), y \( \mathscr{P}(A \cap B) = \{ \emptyset, \{2\} \} \).
        \item Por otro lado, \( \mathscr{P}(A) = \{ \emptyset, \{1\}, \{2\}, \{1, 2\} \} \) y \( \mathscr{P}(B) = \{ \emptyset, \{2\}, \{3\}, \{2, 3\} \} \), y su intersección es \( \mathscr{P}(A) \cap \mathscr{P}(B) = \{ \emptyset, \{2\} \} \).
        \item En este caso, \( \mathscr{P}(A \cap B) = \mathscr{P}(A) \cap \mathscr{P}(B) \).
   \end{itemize}

   La igualdad puede ser verdadera en algunos casos especiales, pero no en general.

    \item \textbf{Segunda proposición:}
   \[
   \mathscr{P} \left( \bigcup_{X \in \mathscr{F}} X \right) = \bigcup_{X \in \mathscr{F}} \mathscr{P}(X)?
   \]
   \textbf{Interpretación:}

   Aquí se está afirmando que el conjunto potencia de la unión de todos los conjuntos en \( \mathscr{F} \) es igual a la unión de los conjuntos potencia de los conjuntos en \( \mathscr{F} \).

   \textbf{Análisis:}

   Esto \textbf{no es cierto}. El conjunto potencia de la unión contiene todos los subconjuntos de la unión, mientras que la unión de los conjuntos potencia solo contiene los subconjuntos que están en al menos uno de los conjuntos. Esta diferencia hace que la igualdad no se mantenga en general.

   Un contraejemplo:
   \begin{itemize}
        \item Sean \( A = \{1\} \) y \( B = \{2\} \).
        \item Entonces, \( A \cup B = \{1, 2\} \), y \( \mathscr{P}(A \cup B) = \{ \emptyset, \{1\}, \{2\}, \{1, 2\} \} \).
        \item Por otro lado, \( \mathscr{P}(A) = \{ \emptyset, \{1\} \} \) y \( \mathscr{P}(B) = \{ \emptyset, \{2\} \} \), y su unión es \( \mathscr{P}(A) \cup \mathscr{P}(B) = \{ \emptyset, \{1\}, \{2\} \} \).
        \item Aquí, \( \mathscr{P}(A \cup B) \neq \mathscr{P}(A) \cup \mathscr{P}(B) \), ya que \( \{1, 2\} \in \mathscr{P}(A \cup B) \), pero no está en la unión de \( \mathscr{P}(A) \) y \( \mathscr{P}(B) \).
   \end{itemize}
\end{enumerate}
\end{solution}
    
    \item En \( \mathbb{R}^2 \) definimos la relación: \( (a, b) R (c, d) \) si y sólo si \( a - c, b - d \in \mathbb{Z} \). Demuestra que esta relación es de equivalencia y calcula su conjunto cociente. (\textbf{Opcional:}) Intenta dibujar las clases de equivalencia respecto a \( R \).

    \begin{solution}
\textbf{P.D:} que \( R \) es una relación de equivalencia:

La relación \( R \) en \( \mathbb{R}^2 \) está definida de la siguiente manera:
\[
(a, b) R (c, d) \iff (a - c) \in \mathbb{Z} \text{ y } (b - d) \in \mathbb{Z}
\]
Para demostrar que \( R \) es una relación de equivalencia, debemos verificar que cumple las tres propiedades de una relación de equivalencia: \textbf{reflexiva}, \textbf{simétrica} y \textbf{transitiva}.

\begin{enumerate}[label=\arabic*.]
    \item \textbf{Reflexividad}:
    Debemos probar que \( (a, b) R (a, b) \) para cualquier \( (a, b) \in \mathbb{R}^2 \).
    
    \[
    a - a = 0 \in \mathbb{Z} \quad \text{y} \quad b - b = 0 \in \mathbb{Z}
    \]
    Por lo tanto, \( (a, b) R (a, b) \), lo que demuestra que la relación es reflexiva.
    \item \textbf{Simetría}:
    Debemos probar que si \( (a, b) R (c, d) \), entonces \( (c, d) R (a, b) \).
    
    Si \( (a, b) R (c, d) \), entonces \( a - c \in \mathbb{Z} \) y \( b - d \in \mathbb{Z} \). Como \( \mathbb{Z} \) es cerrado bajo la negación, tenemos que:
    \[
    c - a = -(a - c) \in \mathbb{Z} \quad \text{y} \quad d - b = -(b - d) \in \mathbb{Z}
    \]
    Por lo tanto, \( (c, d) R (a, b) \), lo que demuestra que la relación es simétrica.
    
    \item \textbf{Transitividad}:
    Debemos probar que si \( (a, b) R (c, d) \) y \( (c, d) R (e, f) \), entonces \( (a, b) R (e, f) \).
    
    Si \( (a, b) R (c, d) \), entonces \( a - c \in \mathbb{Z} \) y \( b - d \in \mathbb{Z} \). Si \( (c, d) R (e, f) \), entonces \( c - e \in \mathbb{Z} \) y \( d - f \in \mathbb{Z} \). Sumando estas ecuaciones, obtenemos:
    \[
    (a - c) + (c - e) = a - e \in \mathbb{Z} \quad \text{y} \quad (b - d) + (d - f) = b - f \in \mathbb{Z}
    \]
    Por lo tanto, \( (a, b) R (e, f) \), lo que demuestra que la relación es transitiva.
\end{enumerate}

\textbf{Conclusión:}
Como la relación \( R \) es reflexiva, simétrica y transitiva, podemos concluir que es una \textbf{relación de equivalencia} en \( \mathbb{R}^2 \).

\textbf{Conjunto cociente:}

El conjunto cociente bajo la relación \( R \) consiste en las clases de equivalencia de los puntos \( (a, b) \in \mathbb{R}^2 \). Dos puntos \( (a, b) \) y \( (c, d) \) están en la misma clase de equivalencia si:
\[
a - c \in \mathbb{Z} \quad \text{y} \quad b - d \in \mathbb{Z}
\]
Esto implica que los puntos \( (a, b) \) y \( (c, d) \) están en la misma clase de equivalencia cada clase de equivalencia corresponde a una traslación entera en ambas coordenadas.

El conjunto cociente \( \mathbb{R}^2 / R \) es isomorfo al \textbf{torno de lados 1}, que es un conjunto que representa los puntos de \( \mathbb{R}^2 \) módulo las traslaciones enteras:
\[
[0, 1) \times [0, 1)
\]
donde cada punto en \( [0, 1) \times [0, 1) \) representa una clase de equivalencia de puntos en \( \mathbb{R}^2 \) bajo la relación \( R \).
\end{solution}
    
    \item En \( \mathbb{R}^2 \) denotamos la circunferencia unitaria por \( S^1 = \{(x, y) \in \mathbb{R}^2 \mid x^2 + y^2 = 1\} \). Definimos, en \( S^1 \), la relación \( P R Q \) si y sólo si \( P = Q \) o \( P \) y \( Q \) son diametralmente opuestos. Demuestra que esta relación es de equivalencia y calcula su conjunto cociente. (\textbf{Opcional:}) Intenta dibujar las clases de equivalencia respecto a \( R \).

    \begin{solution}
\textbf{P.D:} que \( R \) es una relación de equivalencia:

La relación \( R \) en \( S^1 \) está definida como sigue:
\[
P R Q \iff P = Q \text{ o } P \text{ y } Q \text{ son diametralmente opuestos}.
\]
Debemos demostrar que esta relación cumple con las tres propiedades de una relación de equivalencia: \textbf{reflexiva}, \textbf{simétrica} y \textbf{transitiva}.

\begin{enumerate}[label=\arabic*.]
    \item \textbf{Reflexividad:}
    Debemos probar que \( P R P \) para cualquier \( P \in S^1 \).
    
    Esto es cierto porque, por definición, \( P = P \), por lo tanto, la relación es reflexiva.
    
    \item \textbf{Simetría:}
    Debemos probar que si \( P R Q \), entonces \( Q R P \).
    
    Si \( P R Q \), entonces \( P = Q \) o \( P \) y \( Q \) son diametralmente opuestos. Si \( P = Q \), entonces \( Q = P \) y la simetría se cumple. Si \( P \) y \( Q \) son diametralmente opuestos, entonces \( Q \) también es diametralmente opuesto a \( P \), lo que implica que \( Q R P \). Por lo tanto, la relación es simétrica.
    
    \item \textbf{Transitividad:}
    Debemos probar que si \( P R Q \) y \( Q R R \), entonces \( P R R \).
    
    Examinamos dos casos:
    
    \begin{itemize}
        \item Si \( P = Q \) y \( Q = R \), entonces claramente \( P = R \), y por lo tanto \( P R R \).
        \item Si \( P \) y \( Q \) son diametralmente opuestos, y \( Q \) y \( R \) también son diametralmente opuestos, entonces necesariamente \( P = R \), porque los puntos diametralmente opuestos son únicos en \( S^1 \).
    \end{itemize}
    
    Por lo tanto, la relación es transitiva.
\end{enumerate}

\textbf{Conclusión:}
Como la relación \( R \) es reflexiva, simétrica y transitiva, podemos concluir que \( R \) es una \textbf{relación de equivalencia} en \( S^1 \).

\textbf{Conjunto cociente:}

El conjunto cociente bajo la relación \( R \) consiste en las clases de equivalencia de puntos en \( S^1 \). Estas clases de equivalencia agrupan los puntos que son iguales o diametralmente opuestos.

Visualmente, en la circunferencia unitaria \( S^1 \), cada punto \( P \) en la circunferencia está emparejado con su punto diametralmente opuesto \( Q \). Cada clase de equivalencia tiene dos elementos:
\[
[P] = \{P, Q\}
\]
donde \( P \) y \( Q \) son puntos diametralmente opuestos en \( S^1 \).

\textbf{Descripción geométrica del conjunto cociente:}

El conjunto cociente \( S^1 / R \) puede visualizarse como un \textbf{semicírculo}. Al identificar cada punto en \( S^1 \) con su diametralmente opuesto, solo necesitamos considerar los puntos en un semicírculo (por ejemplo, los puntos entre \( 0 \) y \( \pi \) en coordenadas angulares), ya que cada punto tiene un ``gemelo'' diametralmente opuesto en la otra mitad del círculo.

El conjunto cociente \( S^1 / R \) es equivalente a una copia de la \textbf{mitad superior} de la circunferencia unitaria, ya que estamos identificando puntos opuestos en \( S^1 \).
\end{solution}

    \item (\textbf{¡¡¡SuperExtra!!!:}) Sea \( A \) el conjunto de fórmulas de un lenguaje de primer orden. En clase vimos que la relación \( a R b \) si \( a \leftrightarrow b \) es una tautología es una relación de equivalencia. Ahora definimos en \( A/R \) la relación \( [a]_R S[b]_b \) si y sólo si existen representantes \( x \in [a]_R \), \( y \in [b]_R \) tales que \( x \rightarrow y \) es una tautología. Demuestre que \( S \) es un orden para \( A/R \). Además, este orden tiene un elemento mayor y uno menor, ¿cuáles son?

    \begin{solution}
\textbf{P.D:} que \( S \) es un orden para \( A/R \):

La relación \( S \) en \( A/R \) está definida de la siguiente manera:

\[
[a]_R S [b]_R \iff \exists x \in [a]_R, \, y \in [b]_R \text{ tales que } x \rightarrow y \text{ es una tautología}.
\]

Para demostrar que \( S \) es un orden, debemos verificar que cumple las tres propiedades que definen un \textbf{orden parcial}: \textbf{reflexividad}, \textbf{antisimetría} y \textbf{transitividad}.

\begin{enumerate}[label=\arabic*.]
\item \textbf{Reflexividad:}

Queremos probar que \( [a]_R S [a]_R \) para cualquier \( [a]_R \in A/R \).

Por definición de \( S \), esto significa que existe \( x \in [a]_R \) tal que \( x \rightarrow x \) es una tautología. Esto es cierto, ya que \( x \rightarrow x \) es siempre una tautología, independientemente de \( x \). Por lo tanto, \( [a]_R S [a]_R \), y la relación \( S \) es reflexiva.

\item \textbf{Antisimetría:}

Debemos probar que si \( [a]_R S [b]_R \) y \( [b]_R S [a]_R \), entonces \( [a]_R = [b]_R \).

Por definición de \( S \), si \( [a]_R S [b]_R \), existen \( x \in [a]_R \) y \( y \in [b]_R \) tales que \( x \rightarrow y \) es una tautología. De igual forma, si \( [b]_R S [a]_R \), existen \( y \in [b]_R \) y \( x \in [a]_R \) tales que \( y \rightarrow x \) es una tautología. 

Ahora, si \( x \rightarrow y \) y \( y \rightarrow x \) son ambas tautologías, por definición de equivalencia lógica, esto implica que \( x \leftrightarrow y \) es una tautología. Como \( x \in [a]_R \) y \( y \in [b]_R \), esto significa que \( [a]_R = [b]_R \). Por lo tanto, la relación \( S \) es antisimétrica.

\item \textbf{Transitividad:}

Debemos probar que si \( [a]_R S [b]_R \) y \( [b]_R S [c]_R \), entonces \( [a]_R S [c]_R \).

Por definición, si \( [a]_R S [b]_R \), existen \( x \in [a]_R \) y \( y \in [b]_R \) tales que \( x \rightarrow y \) es una tautología. Además, si \( [b]_R S [c]_R \), existen \( y' \in [b]_R \) y \( z \in [c]_R \) tales que \( y' \rightarrow z \) es una tautología. Dado que \( y \in [b]_R \) y \( y' \in [b]_R \), tenemos que \( y \leftrightarrow y' \) es una tautología, lo que implica que \( x \rightarrow z \) es una tautología.

Por lo tanto, \( [a]_R S [c]_R \), y la relación \( S \) es transitiva.
\end{enumerate}

\textbf{Conclusión:}

Dado que la relación \( S \) es reflexiva, antisimétrica y transitiva, podemos concluir que \( S \) es un \textbf{orden parcial} sobre \( A/R \).

\textbf{Elemento mayor y menor en el orden:}

En un orden parcial, el \textbf{elemento mayor} es aquel que está relacionado con todos los demás elementos, y el \textbf{elemento menor} es aquel que está relacionado por todos los demás elementos. En este caso, necesitamos identificar los elementos correspondientes en términos de las tautologías.

\begin{itemize}
    \item El \textbf{elemento mayor} es la clase de equivalencia de la \textbf{tautología}. Esto se debe a que para cualquier fórmula \( a \), \( \top \rightarrow a \) es siempre una tautología. Por lo tanto, la clase de equivalencia \( [\top]_R \) (donde \( \top \) es una tautología) es el mayor elemento.
  
    \item El \textbf{elemento menor} es la clase de equivalencia de la \textbf{contradicción}. Para cualquier fórmula \( a \), \( a \rightarrow \bot \) (donde \( \bot \) es una contradicción) es siempre una tautología. Esto significa que la clase de equivalencia \( [\bot]_R \) (donde \( \bot \) es una contradicción) es el menor elemento.
\end{itemize}

Por lo tanto, el orden definido por \( S \) tiene un elemento mayor \( [\top]_R \) y un elemento menor \( [\bot]_R \).
\end{solution}
\end{enumerate}

\end{document}