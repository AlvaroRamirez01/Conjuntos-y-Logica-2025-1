\documentclass[11pt,answers]{exam}
\usepackage[spanish]{babel}
\usepackage[utf8]{inputenc}
\usepackage[T1]{fontenc}
\usepackage{amsmath,amssymb,amsfonts}
\usepackage{graphicx}
\usepackage{colortbl}
\usepackage{xcolor}
\usepackage{multirow}
\usepackage{float}
\usepackage{enumitem}
\usepackage{algorithm}
\usepackage{mathrsfs}
\usepackage{array} % Para controlar el ancho de las celdas
\usepackage{enumitem} % Para personalizar listas
\usepackage{listings}% http://ctan.org/pkg/listings
\usepackage{hyperref} % for hyperlinks
\usepackage{amsmath} % para matemáticas mejoradas
\usepackage{algorithm} % para escribir pseudocódigos
\usepackage{algpseudocode} % para escribir pseudocódigos
\usepackage[version=4]{mhchem}
\usepackage{stmaryrd}

% \usepackage{etoolbox}

% \AtBeginEnvironment{align}{\setcounter{equation}{0}}

\renewcommand{\solutiontitle}{\noindent\textbf{Solución:}\par\noindent}

\lstset{
  basicstyle=\ttfamily,
  mathescape
}

\graphicspath{{public/}}

\setlength{\topmargin}{-.5in} \setlength{\textheight}{9.25in}
\setlength{\oddsidemargin}{-0.5in} \setlength{\textwidth}{7.2in}


\begin{document}
\begin{center}
    \newcommand{\HRule}{\rule{\linewidth}{0.5mm}}
    \begin{minipage}{0.48\textwidth} 
        \begin{flushleft}
            \includegraphics[scale = 0.08]{../../../public/logo_unam.png}
        \end{flushleft}
    \end{minipage}
    \begin{minipage}{0.48\textwidth} 
        \begin{flushright}
            \includegraphics[scale =0.22]{../../../public/logo_ciencias.png}
        \end{flushright}
    \end{minipage}
    \vspace*{-1.5cm}						
    \textsc{\huge Nacional Autónoma de México \\ \vspace{-4px} Universidad }\\[2cm]	
    \textsc{\LARGE Facultad de Ciencias}\\[1.5cm]
    \vspace*{1cm}					
        \HRule \\[0.7cm]							
            { \huge \bfseries Tarea 03}\\[0.4cm]	
        \HRule \\[1.5cm]						    
    \begin{minipage}{0.52\textwidth}													
        \begin{flushleft} \large	
            \small
            \vspace{-0.6cm}	
            \vspace{-0.6cm}	
                \emph{Alumno:}\\
               Ramírez López Alvaro. 316276355\\
            \vspace*{2cm}
        \end{flushleft}																		
        \end{minipage}		
    \begin{minipage}{0.46\textwidth}		
        \vspace{-0.6cm}											
        \begin{flushright} \large						
            \small										
            \emph{Profesor:} Jesús Villagómez Chávez	\\
            \emph{Ayudantes:}
                Gabriela Peña Franco	 \\
                Martha Rubí Gutiérrez González	 \\
        \end{flushright}																
    \end{minipage}	
    \vspace*{1cm}
    \vspace{2cm}
    \begin{center}						
        {\large 9 de septiembre de 2024}
    \end{center}  						
\end{center}	
\textbf{}
\newpage

Considere para los siguientes ejercicios que $\mathscr{U}$ es el conjunto universo, \( A, B, C \subseteq \mathscr{U} \) y \( R, S \subseteq A \times A \) relaciones binarias sobre \( A \).

\begin{enumerate}
    \item Demuestra que:
    \begin{itemize}
        \item $A \setminus (B \setminus C) = (A \setminus B) \cup (A \cap C)$,
        \item $A \setminus (A \cap B) = A \setminus B$,
        \item $A \setminus B = A \text{ si y sólo si } A \cap B = \emptyset$,
        \item $A \cap B = A \cup B \text{ si y sólo si } A = B$,
        \item $A \setminus (B \setminus C) = (A \setminus B) \setminus C \text{ si y sólo si } A \cap C = \emptyset$.
    \end{itemize}

    \begin{solution}
    \begin{enumerate}
    \item \textbf{El orden en el conjunto de los números racionales} \\
    Utilizamos una relación binaria \( < \) que representa el orden en los números racionales, denotados por \( \mathbb{Q} \). La fórmula en lógica de primer orden es:
    \[
    \forall x, y, z \in \mathbb{Q}, (x < y \land y < z \rightarrow x < z) \land (x < y \rightarrow \neg(y < x))
    \]

    \item \textbf{Que una función sea monótona} \\
    Sea \( f: A \rightarrow B \) una función. La monotonía de una función se puede expresar como:
    \[
    \forall x, y \in A, (x \leq y \rightarrow f(x) \leq f(y))
    \]

    \item \textbf{Que una relación sea de equivalencia} \\
    Sea \( R \) una relación sobre un conjunto \( A \). La relación \( R \) es de equivalencia si es reflexiva, simétrica y transitiva:
    \[
    \text{Reflexiva:} \quad \forall x \in A, R(x, x)
    \]
    \[
    \text{Simétrica:} \quad \forall x, y \in A, (R(x, y) \rightarrow R(y, x))
    \]
    \[
    \text{Transitiva:} \quad \forall x, y, z \in A, (R(x, y) \land R(y, z) \rightarrow R(x, z))
    \]

    \item \textbf{Tener exactamente tres elementos} \\
    Sea \( A \) un conjunto. Para expresar que \( A \) tiene exactamente tres elementos:
    \[
    \exists x, y, z \in A, (x \neq y \land x \neq z \land y \neq z \land \forall w \in A, (w = x \lor w = y \lor w = z))
    \]

    \item \textbf{Nadie en la clase de conjuntos es más inteligente que todos en la clase de lógica} \\
    Supongamos que \( C \) es la clase de conjuntos, \( L \) es la clase de lógica, y \( I(x, y) \) indica que \( x \) es más inteligente que \( y \). La fórmula es:
    \[
    \forall x \in C, \exists y \in L, \neg I(x, y)
    \]
\end{enumerate}
\end{solution}

    \item Demuestra que:
    \begin{itemize}
        \item $\mathscr{P}(A) \subseteq \mathscr{P}(B) \text{ si y sólo si } A \subseteq B$,
        \item $\mathscr{P}(A \cap B) = \mathscr{P}(A) \cap \mathscr{P}(B)$.
        \item ¿Es cierto que \( \mathscr{P}(A \setminus B) = \mathscr{P}(A) \setminus \mathscr{P}(B) \)? En caso negativo, da condiciones sobre \( A \) y \( B \) para que se cumpla la igualdad anterior.
    \end{itemize}

    \begin{solution}
    \begin{itemize}
        \item $\mathscr{P}(A) \subseteq \mathscr{P}(B) \text{ si y sólo si } A \subseteq B$.

        \(\Rightarrow\) Supongamos que \( \mathscr{P}(A) \subseteq \mathscr{P}(B) \). Esto significa que todos los subconjuntos de \( A \) también son subconjuntos de \( B \), es decir, \( \forall X \in \mathscr{P}(A), X \in \mathscr{P}(B) \). En particular, \( A \in \mathscr{P}(A) \) (ya que \( A \) es un subconjunto de sí mismo), lo que implica que \( A \in \mathscr{P}(B) \), es decir, \( A \subseteq B \).

        \(\Leftarrow\) Supongamos ahora que \( A \subseteq B \). Queremos probar que \( \mathscr{P}(A) \subseteq \mathscr{P}(B) \). Dado que \( A \subseteq B \), cualquier subconjunto de \( A \) también es subconjunto de \( B \). Por lo tanto, si \( X \in \mathscr{P}(A) \), entonces \( X \subseteq A \), y como \( A \subseteq B \), se sigue que \( X \subseteq B \), lo que implica que \( X \in \mathscr{P}(B) \). Por lo tanto, \( \mathscr{P}(A) \subseteq \mathscr{P}(B) \).
        
        \[
        \boxed{\mathscr{P}(A) \subseteq \mathscr{P}(B) \iff A \subseteq B}
        \]
        
        \item $\mathscr{P}(A \cap B) = \mathscr{P}(A) \cap \mathscr{P}(B)$.

        \(\subseteq\) Supongamos que \( X \in \mathscr{P}(A \cap B) \). Entonces, \( X \subseteq A \cap B \). Por la definición de intersección, esto significa que \( X \subseteq A \) y \( X \subseteq B \), lo que implica que \( X \in \mathscr{P}(A) \) y \( X \in \mathscr{P}(B) \). Por lo tanto, \( X \in \mathscr{P}(A) \cap \mathscr{P}(B) \).

        \(\supseteq\) Supongamos ahora que \( X \in \mathscr{P}(A) \cap \mathscr{P}(B) \). Esto significa que \( X \in \mathscr{P}(A) \) y \( X \in \mathscr{P}(B) \), es decir, \( X \subseteq A \) y \( X \subseteq B \). Por lo tanto, \( X \subseteq A \cap B \), lo que implica que \( X \in \mathscr{P}(A \cap B) \).
        
        \[
        \boxed{\mathscr{P}(A \cap B) = \mathscr{P}(A) \cap \mathscr{P}(B)}
        \]
        
        \item ¿Es cierto que \( \mathscr{P}(A \setminus B) = \mathscr{P}(A) \setminus \mathscr{P}(B) \)? En caso negativo, da condiciones sobre \( A \) y \( B \) para que se cumpla la igualdad anterior.

        \textbf{Análisis:}

        No es cierto en general. Para que la igualdad \( \mathscr{P}(A \setminus B) = \mathscr{P}(A) \setminus \mathscr{P}(B) \) sea válida, sería necesario que cualquier subconjunto de \( A \setminus B \) estuviera en \( \mathscr{P}(A) \setminus \mathscr{P}(B) \) y viceversa. Sin embargo, esto no siempre es cierto, como veremos con un contraejemplo.
        
        \textbf{Contraejemplo:}
        
        Consideremos \( A = \{1, 2\} \) y \( B = \{2\} \).
        
        \begin{itemize}
            \item \( A \setminus B = \{1\} \).
            \item \( \mathscr{P}(A \setminus B) = \mathscr{P}(\{1\}) = \{\emptyset, \{1\}\} \).
        \end{itemize}
        
        Ahora calculemos \( \mathscr{P}(A) \setminus \mathscr{P}(B) \):
        
        \begin{itemize}
            \item \( \mathscr{P}(A) = \{\emptyset, \{1\}, \{2\}, \{1, 2\}\} \).
            \item \( \mathscr{P}(B) = \{\emptyset, \{2\}\} \).
            \item \( \mathscr{P}(A) \setminus \mathscr{P}(B) = \{\{1\}, \{1, 2\}\} \).
        \end{itemize}
        
        Claramente, \( \mathscr{P}(A \setminus B) = \{\emptyset, \{1\}\} \) \textbf{no es igual} a \( \mathscr{P}(A) \setminus \mathscr{P}(B) = \{\{1\}, \{1, 2\}\} \).
        
        \textbf{Condición para que la igualdad se cumpla:}
        
        La igualdad \( \mathscr{P}(A \setminus B) = \mathscr{P}(A) \setminus \mathscr{P}(B) \) \textbf{se cumple} si y solo si \( B \subseteq A \). Esto es porque, cuando \( B \subseteq A \), ningún subconjunto de \( A \setminus B \) puede estar en \( \mathscr{P}(B) \), y por lo tanto, la diferencia de subconjuntos se comporta de manera consistente.
        
        \[
        \boxed{\mathscr{P}(A \setminus B) = \mathscr{P}(A) \setminus \mathscr{P}(B) \text{ si y sólo si } B \subseteq A}
        \]
    \end{itemize}
\end{solution}
    
    \item Calcula: \(\bigcap \{X \subseteq \mathscr{U} \mid X \in T\} \quad \text{y} \quad \bigcup \{X \subseteq \mathscr{U} \mid X \in T\}\)
    para los siguientes casos:
    \begin{itemize}
        \item \( T = \{\{1, 2\}, \{2, 3\}, \{1, 3\}\} \),
        \item \( T = \mathscr{P}(A) \).
    \end{itemize}

    \begin{solution}
    \begin{itemize}
        \item \( T = \{\{1, 2\}, \{2, 3\}, \{1, 3\}\} \).
        
            Dado \( T = \{\{1, 2\}, \{2, 3\}, \{1, 3\}\} \), necesitamos calcular la \textbf{intersección} y la \textbf{unión} de los conjuntos en \( T \).
            
            \textbf{Intersección:}
            
            La intersección de los conjuntos en \( T \) es el conjunto de elementos que están en todos los subconjuntos de \( T \):
            \[
            \bigcap T = \{1, 2\} \cap \{2, 3\} \cap \{1, 3\}
            \]
            
            Veamos las intersecciones paso a paso:
            \[
            \{1, 2\} \cap \{2, 3\} = \{2\}
            \]
            \[
            \{2\} \cap \{1, 3\} = \emptyset
            \]
            
            Por lo tanto:
            \[
            \bigcap T = \emptyset
            \]
            
            \textbf{Unión:}
            
            La unión de los conjuntos en \( T \) es el conjunto de elementos que pertenecen a al menos uno de los subconjuntos de \( T \):
            \[
            \bigcup T = \{1, 2\} \cup \{2, 3\} \cup \{1, 3\}
            \]
            
            Veamos las uniones paso a paso:
            \[
            \{1, 2\} \cup \{2, 3\} = \{1, 2, 3\}
            \]
            \[
            \{1, 2, 3\} \cup \{1, 3\} = \{1, 2, 3\}
            \]
            
            Por lo tanto:
            \[
            \bigcup T = \{1, 2, 3\}
            \]
            
            Respuesta para \( T = \{\{1, 2\}, \{2, 3\}, \{1, 3\}\} \):
            \[
            \bigcap T = \emptyset
            \]
            \[
            \bigcup T = \{1, 2, 3\}
            \]
        \item \( T = \mathscr{P}(A) \).

        \( T = \mathscr{P}(A) \), donde \( \mathscr{P}(A) \) es la \textbf{familia de subconjuntos} de \( A \)

        La familia \( \mathscr{P}(A) \) es el \textbf{conjunto potencia} de \( A \), es decir, el conjunto de todos los subconjuntos posibles de \( A \). Necesitamos calcular la intersección y la unión de todos los subconjuntos de \( A \).
        
        \textbf{Intersección:}
        
        La intersección de todos los subconjuntos de \( A \), es decir, la intersección de los elementos de \( \mathscr{P}(A) \), es el conjunto de elementos que están presentes en todos los subconjuntos de \( A \).
        
        \begin{itemize}
            \item El único conjunto que está presente en \textbf{todos} los subconjuntos de \( A \) es el conjunto vacío \( \emptyset \). Por lo tanto:
          \[
          \bigcap \mathscr{P}(A) = \emptyset
          \]
        \end{itemize}
        
        \textbf{Unión:}
        
        La unión de todos los subconjuntos de \( A \), es decir, la unión de los elementos de \( \mathscr{P}(A) \), es el conjunto de todos los elementos que aparecen en al menos uno de los subconjuntos de \( A \). Dado que \( \mathscr{P}(A) \) contiene todos los subconjuntos de \( A \), la unión será el conjunto \( A \) mismo:
        \[
        \bigcup \mathscr{P}(A) = A
        \]
        
        Respuesta para \( T = \mathscr{P}(A) \):
        \[
        \bigcap \mathscr{P}(A) = \emptyset
        \]
        \[
        \bigcup \mathscr{P}(A) = A
        \]
    \end{itemize}
\end{solution}

    \item Sean \( \mathscr{F} \) y \( \mathscr{G} \) familias no vacías de subconjuntos de \( \mathscr{U} \). Demuestra que si \( \mathscr{F} \subseteq \mathscr{G} \), entonces:
    \(\bigcup \mathscr{F} \subseteq \bigcup \mathscr{G}.\)

    \begin{solution}
    \textbf{P.D} que si \( \mathscr{F} \subseteq \mathscr{G} \), entonces \( \bigcup \mathscr{F} \subseteq \bigcup \mathscr{G} \).

    Suponemos que \( \mathscr{F} \subseteq \mathscr{G} \). Esto significa que todos los subconjuntos que pertenecen a \( \mathscr{F} \) también pertenecen a \( \mathscr{G} \), es decir, si \( A \in \mathscr{F} \), entonces \( A \in \mathscr{G} \).
    
    Queremos probar que si \( x \in \bigcup \mathscr{F} \), entonces \( x \in \bigcup \mathscr{G} \).
    
    Supongamos que \( x \in \bigcup \mathscr{F} \).
    Por la definición de unión, esto significa que existe un subconjunto \( A \in \mathscr{F} \) tal que \( x \in A \).
    
    Dado que \( \mathscr{F} \subseteq \mathscr{G} \), sabemos que \( A \in \mathscr{G} \). Por lo tanto, \( x \in A \) implica que \( x \in \bigcup \mathscr{G} \) (ya que \( A \) es un subconjunto de \( \mathscr{G} \)).
    
    Conclusión:
    Hemos mostrado que si \( x \in \bigcup \mathscr{F} \), entonces \( x \in \bigcup \mathscr{G} \). Por lo tanto, \( \bigcup \mathscr{F} \subseteq \bigcup \mathscr{G} \).
    
    \[
    \boxed{\bigcup \mathscr{F} \subseteq \bigcup \mathscr{G}}
    \]
\end{solution}

    \item Sea \( \mathscr{F} \) una familia no vacía de subconjuntos de \( \mathscr{U} \). Demuestra que:
    \[
    \left( \bigcap_{X \in \mathscr{F}} X \right)^c = \bigcup_{X \in \mathscr{F}} X^c.
    \]

    \begin{solution}
    
    $\Rightarrow$ Probar que \( \left( \bigcap_{X \in \mathscr{F}} X \right)^c \subseteq \bigcup_{X \in \mathscr{F}} X^c \)
    
    Supongamos que \( x \in \left( \bigcap_{X \in \mathscr{F}} X \right)^c \). Esto significa que \( x \notin \bigcap_{X \in \mathscr{F}} X \), es decir, \( x \) \textbf{no} pertenece a todos los conjuntos \( X \in \mathscr{F} \). Por lo tanto, debe existir al menos un conjunto \( X_0 \in \mathscr{F} \) tal que \( x \notin X_0 \), lo que implica que \( x \in X_0^c \).
    
    Dado que \( x \in X_0^c \), tenemos que \( x \in \bigcup_{X \in \mathscr{F}} X^c \).
    
    Esto demuestra que:
    \[
    \left( \bigcap_{X \in \mathscr{F}} X \right)^c \subseteq \bigcup_{X \in \mathscr{F}} X^c
    \]
    
    $\Leftarrow$ Probar que \( \bigcup_{X \in \mathscr{F}} X^c \subseteq \left( \bigcap_{X \in \mathscr{F}} X \right)^c \)
    
    Supongamos que \( x \in \bigcup_{X \in \mathscr{F}} X^c \). Esto significa que existe al menos un conjunto \( X_0 \in \mathscr{F} \) tal que \( x \in X_0^c \), es decir, \( x \notin X_0 \).
    
    Si \( x \notin X_0 \), entonces \( x \notin \bigcap_{X \in \mathscr{F}} X \), porque la intersección \( \bigcap_{X \in \mathscr{F}} X \) requiere que \( x \) esté en todos los conjuntos de \( \mathscr{F} \), y sabemos que \( x \) no pertenece a \( X_0 \).
    
    Por lo tanto, \( x \in \left( \bigcap_{X \in \mathscr{F}} X \right)^c \).
    
    Esto demuestra que:
    \[
    \bigcup_{X \in \mathscr{F}} X^c \subseteq \left( \bigcap_{X \in \mathscr{F}} X \right)^c
    \]
    
    Conclusión
    
    Dado que hemos demostrado la doble contención:
    \[
    \left( \bigcap_{X \in \mathscr{F}} X \right)^c \subseteq \bigcup_{X \in \mathscr{F}} X^c \quad \text{y} \quad \bigcup_{X \in \mathscr{F}} X^c \subseteq \left( \bigcap_{X \in \mathscr{F}} X \right)^c
    \]
    podemos concluir que:
    \[
    \left( \bigcap_{X \in \mathscr{F}} X \right)^c = \bigcup_{X \in \mathscr{F}} X^c
    \] 
    
\end{solution}
    
    \item Demuestra que:
    \begin{itemize}
        \item $A \times (B \cup C) = (A \times B) \cup (A \times C)$.
        \item $(A \cap B) \times C = (A \times C) \cap (B \times C)$.
        \item $(A \times B)^c = (A^c \times B) \cup (A \times B^c) \cup (A^c \times B^c)$.
    \end{itemize}
        
    \begin{solution}
    \begin{itemize}
        \item $A \times (B \cup C) = (A \times B) \cup (A \times C)$.

        \textbf{Demostración:}

        \textbf{Lado izquierdo}: El producto cartesiano \( A \times (B \cup C) \) es el conjunto de todos los pares \( (a, x) \) donde \( a \in A \) y \( x \in B \cup C \), es decir:
          \[
          A \times (B \cup C) = \{(a, x) \mid a \in A, x \in B \cup C\}
          \]
          Como \( x \in B \cup C \), esto significa que \( x \in B \) o \( x \in C \), por lo que:
          \[
          A \times (B \cup C) = \{(a, x) \mid a \in A \text{ y } (x \in B \text{ o } x \in C)\}
          \]
        
        \textbf{Lado derecho}: El conjunto \( (A \times B) \cup (A \times C) \) es la unión de los pares en \( A \times B \) con los pares en \( A \times C \), es decir:
          \[
          (A \times B) \cup (A \times C) = \{(a, x) \mid (a \in A \text{ y } x \in B) \text{ o } (a \in A \text{ y } x \in C)\}
          \]
          Esto también representa el conjunto de todos los pares \( (a, x) \) donde \( a \in A \) y \( x \in B \cup C \).
        
        Como ambos lados describen los mismos elementos, tenemos que:
        \[
        A \times (B \cup C) = (A \times B) \cup (A \times C)
        \]
        \item $(A \cap B) \times C = (A \times C) \cap (B \times C)$.

        \textbf{Demostración:}
        
        \textbf{Lado izquierdo:} El producto cartesiano \( (A \cap B) \times C \) es el conjunto de todos los pares \( (x, y) \) donde \( x \in A \cap B \) y \( y \in C \), es decir:
          \[
          (A \cap B) \times C = \{(x, y) \mid x \in A \cap B, y \in C\}
          \]
          Dado que \( x \in A \cap B \), esto significa que \( x \in A \) y \( x \in B \). Entonces podemos reescribirlo como:
          \[
          (A \cap B) \times C = \{(x, y) \mid x \in A, x \in B, y \in C\}
          \]
        
        \textbf{Lado derecho:} El conjunto \( (A \times C) \cap (B \times C) \) es la intersección de los pares en \( A \times C \) con los pares en \( B \times C \). Es decir:
          \[
          (A \times C) \cap (B \times C) = \{(x, y) \mid (x \in A \text{ y } y \in C) \text{ y } (x \in B \text{ y } y \in C)\}
          \]
          Esto implica que \( x \in A \cap B \) y \( y \in C \), lo que es equivalente a \( (A \cap B) \times C \).
        
        Por lo tanto, tenemos que:
        \[
        (A \cap B) \times C = (A \times C) \cap (B \times C)
        \]
        \item $(A \times B)^c = (A^c \times B) \cup (A \times B^c) \cup (A^c \times B^c)$.

        \textbf{Demostración:}

        (\textbf{Lado izquierdo:} El complemento de \( A \times B \), es decir, \( (A \times B)^c \), en el producto cartesiano \( \mathscr{U} \times \mathscr{V} \) (donde \( A \subseteq \mathscr{U} \) y \( B \subseteq \mathscr{V} \)) está dado por el conjunto de todos los pares \( (x, y) \) que **no** están en \( A \times B \), es decir:
          \[
          (A \times B)^c = \{(x, y) \mid (x, y) \notin A \times B\}
          \]
          Esto significa que \( x \notin A \) o \( y \notin B \). Lo podemos descomponer en tres posibilidades:
          \begin{itemize}
            \item \( x \notin A \) y \( y \in B \),
            \item \( x \in A \) y \( y \notin B \),
            \item \( x \notin A \) y \( y \notin B \).
          \end{itemize}
        
          Entonces:
          \[
          (A \times B)^c = \{(x, y) \mid x \in A^c \text{ y } y \in B\} \cup \{(x, y) \mid x \in A \text{ y } y \in B^c\} \cup \{(x, y) \mid x \in A^c \text{ y } y \in B^c\}
          \]
        
        (\textbf{Lado derecho}: El conjunto \( (A^c \times B) \cup (A \times B^c) \cup (A^c \times B^c) \) es la unión de:
        \begin{itemize}
            \item Pares \( (x, y) \) donde \( x \in A^c \) y \( y \in B \),
            \item Pares \( (x, y) \) donde \( x \in A \) y \( y \in B^c \),
            \item Pares \( (x, y) \) donde \( x \in A^c \) y \( y \in B^c \).
        \end{itemize}
        
        Esto es exactamente la misma descripción del complemento de \( A \times B \). Por lo tanto:
        \[
        (A \times B)^c = (A^c \times B) \cup (A \times B^c) \cup (A^c \times B^c)
        \]
    \end{itemize}
\end{solution}

    \item Define 6 relaciones sobre \( A = \{1, 2\} \).

    \begin{solution}
    Para definir relaciones sobre \( A = \{1, 2\} \), definimos nuestro conjunto asi: 
    
    \( A \times A = \{(1, 1), (1, 2), (2, 1), (2, 2)\} \).

    A continuación se presentan seis posibles relaciones sobre \( A \):

    \begin{enumerate}
        \item Relación vacía: No hay pares en la relación.  
        \( R_1 = \emptyset \)

        \item Relación identidad: Solo se relacionan los elementos consigo mismos.  
        \( R_2 = \{(1, 1), (2, 2)\} \)

        \item Relación completa: Todos los pares posibles están en la relación.  
           \( R_3 = \{(1, 1), (1, 2), (2, 1), (2, 2)\} \)
        
        \item Relación asimétrica: Relaciona \( 1 \) con \( 2 \), pero no viceversa.  
           \( R_4 = \{(1, 2)\} \)
        
        \item Relación inversa de la anterior: Relaciona \( 2 \) con \( 1 \), pero no viceversa.  
           \( R_5 = \{(2, 1)\} \)
        
        \item Relación trivial: Relaciona \( 1 \) consigo mismo y \( 2 \) con \( 1 \).  
           \( R_6 = \{(1, 1), (2, 1)\} \)
    \end{enumerate}
\end{solution}

    \item Describe la relación de equivalencia sobre \( A = \{1, 2, 3, 4, 5\} \) determinada por la partición \( \mathscr{F} = \{\{1, 3\}, \{2, 4\}, \{5\}\} \).

    \begin{solution}
\begin{itemize}
    \item El conjunto es \( A = \{1, 2, 3, 4, 5\} \).
    \item La partición es \( \mathscr{F} = \{\{1, 3\}, \{2, 4\}, \{5\}\} \).
\end{itemize}    
    
    La relación de equivalencia correspondiente a esta partición es aquella en la que dos elementos de \( A \) están relacionados si y solo si están en el mismo subconjunto de \( \mathscr{F} \). Entonces, describimos la relación \( R \) como un conjunto de pares ordenados de \( A \) tal que:
    
    \begin{itemize}
        \item Elementos en \( \{1, 3\} \) están relacionados entre sí:
       \[
       (1, 1), (1, 3), (3, 1), (3, 3)
       \]
    
        \item Elementos en \( \{2, 4\} \) están relacionados entre sí:
       \[
       (2, 2), (2, 4), (4, 2), (4, 4)
       \]
    
        \item Elemento en \( \{5\} \) está relacionado solo consigo mismo:
       \[
       (5, 5)
       \]
    
    \end{itemize}
    
    \textbf{Relación de equivalencia completa:}
    La relación de equivalencia \( R \) asociada a la partición \( \mathscr{F} \) es:
    \[
    R = \{(1, 1), (1, 3), (3, 1), (3, 3), (2, 2), (2, 4), (4, 2), (4, 4), (5, 5)\}
    \]
    
    Esta relación agrupa los elementos según la partición dada, relacionando elementos que están en el mismo subconjunto y, por tanto, determina la relación de equivalencia que respeta la partición \( \mathscr{F} \).
\end{solution}

    \item Considera que \( A \) es un conjunto de 7 elementos y que \( R \) y \( S \) son relaciones de equivalencia. Si \( R \) induce 7 clases de equivalencia y \( S \) una, ¿qué relaciones son \( R \) y \( S \)?

    \begin{solution}

\textbf{Relación \( R \):}

Sabemos que \( R \) induce 7 clases de equivalencia. Esto significa que cada elemento de \( A \) está en su propia clase de equivalencia, es decir, no hay dos elementos que estén relacionados bajo \( R \) salvo consigo mismos. En otras palabras, la relación \( R \) es la \textbf{relación identidad}. Formalmente, esto se escribe como:
\[
R = \{(a, a) \mid a \in A\}
\]
Esta relación solo relaciona cada elemento consigo mismo y no relaciona elementos distintos. Por lo tanto, \( R \) es la relación más fina posible, donde cada clase de equivalencia contiene exactamente un elemento.

\textbf{Relación \( S \):}

Por otro lado, sabemos que \( S \) induce una sola clase de equivalencia. Esto significa que todos los elementos de \( A \) están en la misma clase de equivalencia bajo \( S \). En otras palabras, la relación \( S \) es la \textbf{relación total}. Formalmente, esto se escribe como:
\[
S = A \times A = \{(a, b) \mid a, b \in A\}
\]
En esta relación, cualquier par de elementos \( a \) y \( b \) están relacionados, por lo que solo hay una clase de equivalencia que contiene a todos los elementos de \( A \).

\end{solution}

    \item Determina si en cada caso la relación definida es de equivalencia. En tal caso, describe su partición inducida.
    \begin{itemize}
        \item \( n R m \) en \( \mathbb{Z} \) si \( nm \geq 0 \),
        \item \( a R b \) en \( \mathbb{Z} \) si \( a + b \) es par,
        \item \( a R b \) en \( \mathbb{Z} \) si \( a + b \) es impar,
        \item \( a R b \) en \( \mathbb{Z} \) si \( a^2 + a = b^2 + b \),
        \item \( x R y \) en \( \mathbb{R} \) si \( |x| = |y| \),
        \item \( x R y \) en \( \mathbb{R} \) si \( x^2 + y^2 = 4 \),
        \item \( (x, y) R (x', y') \) en \( \mathbb{R}^2 \) si \( y = y' \),
        \item \( x R y \) en \( \mathbb{R} \) si \( |x - y| \leq 4 \),
        \item Considera \( M \) el conjunto de todos los meses de este año. \( a R b \) en \( M \) si \( a \) y \( b \) comienzan el mismo día de la semana,
        \item \( a R b \) en \( \mathbb{Z} \) si \( a = b \) o \( a + b = 3 \).
    \end{itemize}

    \begin{solution}
    \begin{enumerate}
        \item \( n R m \) en \( \mathbb{Z} \) si \( nm \geq 0 \).

        \textbf{No es relación de equivalencia.} No es simétrica, ya que si \( n = 0 \), entonces \( nR m \) para cualquier \( m \), pero \( mR n \) no necesariamente se cumple para \( m \neq 0 \).
        \item \( a R b \) en \( \mathbb{Z} \) si \( a + b \) es par.
        
        \textbf{Es relación de equivalencia.}
          \begin{itemize}
                \item Reflexiva: \( a + a = 2a \), que es par.
                \item Simétrica: Si \( a + b \) es par, entonces \( b + a \) también lo es.
                \item Transitiva: Si \( a + b \) y \( b + c \) son pares, entonces \( a + c \) también lo es.
          \end{itemize}
        Partición: Números pares y números impares.   
        \item \( a R b \) en \( \mathbb{Z} \) si \( a + b \) es impar.
        
        \textbf{No es relación de equivalencia.} No es reflexiva, ya que \( a + a \) es siempre par, no impar.
        
        \item \( a R b \) en \( \mathbb{Z} \) si \( a^2 + a = b^2 + b \).

        \textbf{Es relación de equivalencia.}
          \begin{itemize}
            \item Reflexiva: \( a^2 + a = a^2 + a \).
            \item Simétrica: Si \( a^2 + a = b^2 + b \), entonces \( b^2 + b = a^2 + a \).
            \item Transitiva: Si \( a^2 + a = b^2 + b \) y \( b^2 + b = c^2 + c \), entonces \( a^2 + a = c^2 + c \).
          \end{itemize}
        Partición: Agrupa números según sus valores de \( n(n + 1) \), como \( \{0\}, \{-1, 0\}, \{2, -3\}, \{3, -4\}, \dots \).
        
        \item \( x R y \) en \( \mathbb{R} \) si \( |x| = |y| \).

        \textbf{Es relación de equivalencia.}
          \begin{itemize}
            \item Reflexiva: \( |x| = |x| \).
            \item Simétrica: Si \( |x| = |y| \), entonces \( |y| = |x| \).
            \item Transitiva: Si \( |x| = |y| \) y \( |y| = |z| \), entonces \( |x| = |z| \).
          \end{itemize}
        Partición: Agrupa números en pares de opuestos: \( \{x, -x\} \).
        
        \item \( x R y \) en \( \mathbb{R} \) si \( x^2 + y^2 = 4 \).

        \textbf{No es relación de equivalencia.} No es transitiva. Por ejemplo, \( (2, 0) \) y \( (0, 2) \) están relacionados con \( (0, 2) \) y \( (2, 0) \), pero no entre sí.
        
        \item \( (x, y) R (x', y') \) en \( \mathbb{R}^2 \) si \( y = y' \).

        \textbf{Es relación de equivalencia.}
          \begin{itemize}
            \item Reflexiva: \( y = y \).
            \item Simétrica: Si \( y = y' \), entonces \( y' = y \).
            \item Transitiva: Si \( y = y' \) y \( y' = y'' \), entonces \( y = y'' \).
          \end{itemize}
        Partición: Agrupa puntos por coordenada \( y \), es decir, las líneas horizontales en \( \mathbb{R}^2 \).
        \item \( x R y \) en \( \mathbb{R} \) si \( |x - y| \leq 4 \).

        \textbf{No es relación de equivalencia.} No es transitiva. Por ejemplo, \( |x - y| \leq 4 \) y \( |y - z| \leq 4 \) no implica que \( |x - z| \leq 4 \).
        
        \item Considera \( M \) el conjunto de todos los meses de este año. \( a R b \) en \( M \) si \( a \) y \( b \) comienzan el mismo día de la semana.
        
        \textbf{Es relación de equivalencia.}
          \begin{itemize}
            \item Reflexiva: Cada mes comienza el mismo día que sí mismo.
            \item Simétrica: Si \( a \) y \( b \) comienzan el mismo día, entonces \( b \) y \( a \) también.
            \item Transitiva: Si \( a \) y \( b \), y \( b \) y \( c \) comienzan el mismo día, entonces \( a \) y \( c \) también.
          \end{itemize}
        Partición: Agrupa los meses según su día de inicio de la semana.

        \item \( a R b \) en \( \mathbb{Z} \) si \( a = b \) o \( a + b = 3 \).
        
        \textbf{No es relación de equivalencia.} No es transitiva. Por ejemplo, \( 1 R 2 \) y \( 2 R 1 \), pero \( 1 R 1 \) no se cumple bajo la condición \( a + b = 3 \).
    \end{enumerate}
\end{solution}

    \item Supón que \( R \) y \( S \) son relaciones de equivalencia. Demuestra que si \( A/R = A/S \), entonces \( R = S \).

    \begin{solution}
    P.D: \( A/R = A/S \) entonces $R=S$.
    
    $\Rightarrow$ Demostrar que \( R \subseteq S \)
    
    Queremos probar que si \( (a, b) \in R \), entonces \( (a, b) \in S \).
    
    Supongamos que \( (a, b) \in R \), lo que significa que \( a \) y \( b \) pertenecen a la misma clase de equivalencia bajo \( R \), es decir, \( b \in [a]_R \). Dado que sabemos que \( A/R = A/S \), las clases de equivalencia de \( R \) y \( S \) son iguales. Por lo tanto:
    \[
    [a]_R = [a]_S
    \]
    Esto implica que \( b \in [a]_S \), lo que significa que \( (a, b) \in S \).
    
    Hemos demostrado que si \( (a, b) \in R \), entonces \( (a, b) \in S \), lo que implica que \( R \subseteq S \).
    
    $\Leftarrow$ Demostrar que \( S \subseteq R \)
    
    Ahora queremos probar que si \( (a, b) \in S \), entonces \( (a, b) \in R \).
    
    Supongamos que \( (a, b) \in S \), lo que significa que \( b \in [a]_S \). Dado que \( A/R = A/S \), las clases de equivalencia bajo \( S \) y \( R \) son iguales, es decir:
    \[
    [a]_S = [a]_R
    \]
    Por lo tanto, \( b \in [a]_R \), lo que implica que \( (a, b) \in R \).
    
    Conclusión
    Hemos demostrado que si \( (a, b) \in S \), entonces \( (a, b) \in R \), lo que implica que \( S \subseteq R \).
    
    Conclusión final
    
    Dado que hemos demostrado que \( R \subseteq S \) y \( S \subseteq R \), podemos concluir que \( R = S \).
    
    \[
    R \subseteq S \quad \text{y} \quad S \subseteq R \implies R = S
    \]
    
    Por lo tanto, si \( A/R = A/S \), entonces \( R = S \). \(\blacksquare\)
\end{solution}

    \item Demuestra que las siguientes relaciones son órdenes parciales, dibuja su diagrama de Hasse e identifica sus elementos mínimos, máximos, menores y mayores.
    \begin{enumerate}
        \item La divisibilidad de los naturales,
        \item El menor igual de los enteros,
        \item Si \( A = \{1, 2, 3, 4\} \), la contención sobre los subconjuntos no vacíos de \( A \),
        \item Si \( A = \{1, 2, 3, 4\} \), la contención sobre los subconjuntos propios de \( A \).
    \end{enumerate}

    \begin{solution}
    \begin{enumerate}
        \item La divisibilidad de los naturales.
        
        \textbf{Relación:}
        Sea \( a, b \in \mathbb{N} \), decimos que \( a \leq b \) si \( a \) divide a \( b \) (denotado como \( a \mid b \)).
        
        \textbf{Propiedades de orden parcial:}
        \begin{itemize}
            \item Reflexividad: Todo número natural divide a sí mismo, es decir, \( a \mid a \) para todo \( a \in \mathbb{N} \).
            \item Antisimetría: Si \( a \mid b \) y \( b \mid a \), entonces \( a = b \). Esto se sigue de la propiedad de divisibilidad.
            \item Transitividad: Si \( a \mid b \) y \( b \mid c \), entonces \( a \mid c \). Esta es la transitividad de la divisibilidad.
        \end{itemize}        
        Por lo tanto, la divisibilidad es un orden parcial.
        
        \textbf{Diagrama de Hasse:}
        Consideremos un conjunto pequeño de números naturales: \( \{1, 2, 4, 8, 16\} \).
        
        El diagrama de Hasse es:
        \begin{verbatim}
           16
            |
            8
            |
            4
            |
            2
            |
            1
        \end{verbatim}
        
        \textbf{Elementos mínimos y máximos:}
        \begin{itemize}
            \item Mínimo: \( 1 \) es el elemento mínimo porque divide a todos los demás.
            \item Máximo: \( 16 \) es el elemento máximo porque es divisible por todos los demás.
        \end{itemize}
        
        \textbf{Menores y mayores:}
        \begin{itemize}
            \item \( 1 \) es menor que todos los demás.
            \item \( 16 \) es mayor que todos los demás.
            \item \( 4 \) es mayor que \( 2 \), y menor que \( 8 \) y \( 16 \).
        \end{itemize}
        
        \item El menor igual de los enteros.

        \textbf{Relación:}
        Sea \( a, b \in \mathbb{Z} \), decimos que \( a \leq b \) si \( a \) es menor o igual que \( b \).
        
        \textbf{Propiedades de orden parcial:}
        \begin{itemize}
            \item Reflexividad: \( a \leq a \) para todo \( a \in \mathbb{Z} \).
            \item Antisimetría: Si \( a \leq b \) y \( b \leq a \), entonces \( a = b \).
            \item Transitividad: Si \( a \leq b \) y \( b \leq c \), entonces \( a \leq c \).
        \end{itemize}
        
        Por lo tanto, el menor igual es un orden parcial.
        
        \textbf{Diagrama de Hasse:}
        Consideremos un subconjunto de enteros \( \{-1, 0, 1, 2\} \).
        
        El diagrama de Hasse es:
        \begin{verbatim}
            2
            |
            1
            |
            0
            |
           -1
        \end{verbatim}
        
        \textbf{Elementos mínimos y máximos:}
        \begin{itemize}
            \item Mínimo: \( -1 \) es el mínimo.
            \item Máximo: \( 2 \) es el máximo.
        \end{itemize}
        
        \textbf{Menores y mayores:}
        \begin{itemize}
            \item \( -1 \) es menor que todos los demás.
            \item \( 2 \) es mayor que todos los demás.
            \item \( 1 \) es mayor que \( 0 \) y menor que \( 2 \).
        \end{itemize}
        
        \item Si \( A = \{1, 2, 3, 4\} \), la contención sobre los subconjuntos no vacíos de \( A \).

        \textbf{Relación:}
        La relación es la contención \( \subseteq \) entre los subconjuntos no vacíos de \( A \).
        
        Los subconjuntos no vacíos de \( A \) son:
        \[
        \{1\}, \{2\}, \{3\}, \{4\}, \{1, 2\}, \{1, 3\}, \{1, 4\}, \dots, \{1, 2, 3, 4\}
        \]
        
        \textbf{Propiedades de orden parcial:}
        \begin{itemize}
            \item Reflexividad: Todo subconjunto está contenido en sí mismo.
            \item Antisimetría: Si \( X \subseteq Y \) y \( Y \subseteq X \), entonces \( X = Y \).
            \item Transitividad: Si \( X \subseteq Y \) y \( Y \subseteq Z \), entonces \( X \subseteq Z \).
        \end{itemize}
        
        Por lo tanto, la contención es un \textbf{orden parcial}.
        
        % \textbf{Diagrama de Hasse:}
        % Consideremos los subconjuntos \( \{1\}, \{1, 2\}, \{1, 2, 3\}, \{1, 2, 3, 4\} \).
        
        % El diagrama de Hasse es:
        % \begin{verbatim}
        %    {1, 2, 3, 4}
        %       |
        %    {1, 2, 3}
        %       |
        %    {1, 2} 
        %       |
        %    {1}
        % \end{verbatim}
        
        \textbf{Elementos mínimos y máximos:}
        \begin{itemize}
            \item Mínimos: Los subconjuntos con un solo elemento, como \( \{1\}, \{2\}, \{3\}, \{4\} \).
            \item Máximo: El conjunto \( A = \{1, 2, 3, 4\} \) es el máximo.
        \end{itemize}
        
        \textbf{Menores y mayores:}
        \( \{1\} \) es menor que \( \{1, 2\} \), que a su vez es menor que \( \{1, 2, 3\} \), y así sucesivamente.
        
        \item Si \( A = \{1, 2, 3, 4\} \), la contención sobre los subconjuntos propios de \( A \).

        \textbf{Relación:}
        La relación es la contención \( \subseteq \) entre los subconjuntos \textbf{propios} de \( A \), es decir, aquellos subconjuntos que no son iguales a \( A \).
        
        Los subconjuntos propios de \( A \) son:
        \[
        \{1\}, \{2\}, \{3\}, \{4\}, \{1, 2\}, \{1, 3\}, \{1, 4\}, \dots, \{1, 2, 3\}
        \]
        
        \textbf{Propiedades de orden parcial:}
        \begin{itemize}
            \item Reflexividad: Todo subconjunto está contenido en sí mismo.
            \item Antisimetría: Si \( X \subseteq Y \) y \( Y \subseteq X \), entonces \( X = Y \).
            \item Transitividad: Si \( X \subseteq Y \) y \( Y \subseteq Z \), entonces \( X \subseteq Z \).
        \end{itemize}
        
        Por lo tanto, la contención es un orden parcial.
        
        % \textbf{Diagrama de Hasse:}
        % Consideremos los subconjuntos \( \{1\}, \{1, 2\}, \{1, 2, 3\} \).
        
        % El diagrama de Hasse es:
        % \begin{verbatim}
        %    {1, 2, 3}
        %       |
        %    {1, 2}
        %       |
        %    {1}
        % \end{verbatim}
        
        \textbf{Elementos mínimos y máximos:}
        \begin{itemize}
            \item Mínimos: Los subconjuntos con un solo elemento, como \( \{1\}, \{2\}, \{3\}, \{4\} \).
            \item Máximo: El conjunto \( \{1, 2, 3\} \) es el máximo subconjunto propio.
        \end{itemize}
        
        \textbf{Menores y mayores:}
        
        Los subconjuntos más pequeños están contenidos en los más grandes.
    \end{enumerate}
\end{solution}
    
    
\end{enumerate}

\end{document}