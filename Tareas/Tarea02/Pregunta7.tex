\begin{solution}
    Para definir relaciones sobre \( A = \{1, 2\} \), definimos nuestro conjunto asi: 
    
    \( A \times A = \{(1, 1), (1, 2), (2, 1), (2, 2)\} \).

    A continuación se presentan seis posibles relaciones sobre \( A \):

    \begin{enumerate}
        \item Relación vacía: No hay pares en la relación.  
        \( R_1 = \emptyset \)

        \item Relación identidad: Solo se relacionan los elementos consigo mismos.  
        \( R_2 = \{(1, 1), (2, 2)\} \)

        \item Relación completa: Todos los pares posibles están en la relación.  
           \( R_3 = \{(1, 1), (1, 2), (2, 1), (2, 2)\} \)
        
        \item Relación asimétrica: Relaciona \( 1 \) con \( 2 \), pero no viceversa.  
           \( R_4 = \{(1, 2)\} \)
        
        \item Relación inversa de la anterior: Relaciona \( 2 \) con \( 1 \), pero no viceversa.  
           \( R_5 = \{(2, 1)\} \)
        
        \item Relación trivial: Relaciona \( 1 \) consigo mismo y \( 2 \) con \( 1 \).  
           \( R_6 = \{(1, 1), (2, 1)\} \)
    \end{enumerate}
\end{solution}