\begin{solution}
    \begin{itemize}
        \item \( T = \{\{1, 2\}, \{2, 3\}, \{1, 3\}\} \).
        
            Dado \( T = \{\{1, 2\}, \{2, 3\}, \{1, 3\}\} \), necesitamos calcular la \textbf{intersección} y la \textbf{unión} de los conjuntos en \( T \).
            
            \textbf{Intersección:}
            
            La intersección de los conjuntos en \( T \) es el conjunto de elementos que están en todos los subconjuntos de \( T \):
            \[
            \bigcap T = \{1, 2\} \cap \{2, 3\} \cap \{1, 3\}
            \]
            
            Veamos las intersecciones paso a paso:
            \[
            \{1, 2\} \cap \{2, 3\} = \{2\}
            \]
            \[
            \{2\} \cap \{1, 3\} = \emptyset
            \]
            
            Por lo tanto:
            \[
            \bigcap T = \emptyset
            \]
            
            \textbf{Unión:}
            
            La unión de los conjuntos en \( T \) es el conjunto de elementos que pertenecen a al menos uno de los subconjuntos de \( T \):
            \[
            \bigcup T = \{1, 2\} \cup \{2, 3\} \cup \{1, 3\}
            \]
            
            Veamos las uniones paso a paso:
            \[
            \{1, 2\} \cup \{2, 3\} = \{1, 2, 3\}
            \]
            \[
            \{1, 2, 3\} \cup \{1, 3\} = \{1, 2, 3\}
            \]
            
            Por lo tanto:
            \[
            \bigcup T = \{1, 2, 3\}
            \]
            
            Respuesta para \( T = \{\{1, 2\}, \{2, 3\}, \{1, 3\}\} \):
            \[
            \bigcap T = \emptyset
            \]
            \[
            \bigcup T = \{1, 2, 3\}
            \]
        \item \( T = \mathscr{P}(A) \).

        \( T = \mathscr{P}(A) \), donde \( \mathscr{P}(A) \) es la \textbf{familia de subconjuntos} de \( A \)

        La familia \( \mathscr{P}(A) \) es el \textbf{conjunto potencia} de \( A \), es decir, el conjunto de todos los subconjuntos posibles de \( A \). Necesitamos calcular la intersección y la unión de todos los subconjuntos de \( A \).
        
        \textbf{Intersección:}
        
        La intersección de todos los subconjuntos de \( A \), es decir, la intersección de los elementos de \( \mathscr{P}(A) \), es el conjunto de elementos que están presentes en todos los subconjuntos de \( A \).
        
        \begin{itemize}
            \item El único conjunto que está presente en \textbf{todos} los subconjuntos de \( A \) es el conjunto vacío \( \emptyset \). Por lo tanto:
          \[
          \bigcap \mathscr{P}(A) = \emptyset
          \]
        \end{itemize}
        
        \textbf{Unión:}
        
        La unión de todos los subconjuntos de \( A \), es decir, la unión de los elementos de \( \mathscr{P}(A) \), es el conjunto de todos los elementos que aparecen en al menos uno de los subconjuntos de \( A \). Dado que \( \mathscr{P}(A) \) contiene todos los subconjuntos de \( A \), la unión será el conjunto \( A \) mismo:
        \[
        \bigcup \mathscr{P}(A) = A
        \]
        
        Respuesta para \( T = \mathscr{P}(A) \):
        \[
        \bigcap \mathscr{P}(A) = \emptyset
        \]
        \[
        \bigcup \mathscr{P}(A) = A
        \]
    \end{itemize}
\end{solution}