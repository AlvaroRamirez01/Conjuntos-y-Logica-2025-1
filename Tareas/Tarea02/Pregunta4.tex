\begin{solution}
    \textbf{P.D} que si \( \mathscr{F} \subseteq \mathscr{G} \), entonces \( \bigcup \mathscr{F} \subseteq \bigcup \mathscr{G} \).

    Suponemos que \( \mathscr{F} \subseteq \mathscr{G} \). Esto significa que todos los subconjuntos que pertenecen a \( \mathscr{F} \) también pertenecen a \( \mathscr{G} \), es decir, si \( A \in \mathscr{F} \), entonces \( A \in \mathscr{G} \).
    
    Queremos probar que si \( x \in \bigcup \mathscr{F} \), entonces \( x \in \bigcup \mathscr{G} \).
    
    Supongamos que \( x \in \bigcup \mathscr{F} \).
    Por la definición de unión, esto significa que existe un subconjunto \( A \in \mathscr{F} \) tal que \( x \in A \).
    
    Dado que \( \mathscr{F} \subseteq \mathscr{G} \), sabemos que \( A \in \mathscr{G} \). Por lo tanto, \( x \in A \) implica que \( x \in \bigcup \mathscr{G} \) (ya que \( A \) es un subconjunto de \( \mathscr{G} \)).
    
    Conclusión:
    Hemos mostrado que si \( x \in \bigcup \mathscr{F} \), entonces \( x \in \bigcup \mathscr{G} \). Por lo tanto, \( \bigcup \mathscr{F} \subseteq \bigcup \mathscr{G} \).
    
    \[
    \boxed{\bigcup \mathscr{F} \subseteq \bigcup \mathscr{G}}
    \]
\end{solution}