\begin{solution}
\begin{itemize}
    \item El conjunto es \( A = \{1, 2, 3, 4, 5\} \).
    \item La partición es \( \mathscr{F} = \{\{1, 3\}, \{2, 4\}, \{5\}\} \).
\end{itemize}    
    
    La relación de equivalencia correspondiente a esta partición es aquella en la que dos elementos de \( A \) están relacionados si y solo si están en el mismo subconjunto de \( \mathscr{F} \). Entonces, describimos la relación \( R \) como un conjunto de pares ordenados de \( A \) tal que:
    
    \begin{itemize}
        \item Elementos en \( \{1, 3\} \) están relacionados entre sí:
       \[
       (1, 1), (1, 3), (3, 1), (3, 3)
       \]
    
        \item Elementos en \( \{2, 4\} \) están relacionados entre sí:
       \[
       (2, 2), (2, 4), (4, 2), (4, 4)
       \]
    
        \item Elemento en \( \{5\} \) está relacionado solo consigo mismo:
       \[
       (5, 5)
       \]
    
    \end{itemize}
    
    \textbf{Relación de equivalencia completa:}
    La relación de equivalencia \( R \) asociada a la partición \( \mathscr{F} \) es:
    \[
    R = \{(1, 1), (1, 3), (3, 1), (3, 3), (2, 2), (2, 4), (4, 2), (4, 4), (5, 5)\}
    \]
    
    Esta relación agrupa los elementos según la partición dada, relacionando elementos que están en el mismo subconjunto y, por tanto, determina la relación de equivalencia que respeta la partición \( \mathscr{F} \).
\end{solution}