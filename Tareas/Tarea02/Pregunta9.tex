\begin{solution}

\textbf{Relación \( R \):}

Sabemos que \( R \) induce 7 clases de equivalencia. Esto significa que cada elemento de \( A \) está en su propia clase de equivalencia, es decir, no hay dos elementos que estén relacionados bajo \( R \) salvo consigo mismos. En otras palabras, la relación \( R \) es la \textbf{relación identidad}. Formalmente, esto se escribe como:
\[
R = \{(a, a) \mid a \in A\}
\]
Esta relación solo relaciona cada elemento consigo mismo y no relaciona elementos distintos. Por lo tanto, \( R \) es la relación más fina posible, donde cada clase de equivalencia contiene exactamente un elemento.

\textbf{Relación \( S \):}

Por otro lado, sabemos que \( S \) induce una sola clase de equivalencia. Esto significa que todos los elementos de \( A \) están en la misma clase de equivalencia bajo \( S \). En otras palabras, la relación \( S \) es la \textbf{relación total}. Formalmente, esto se escribe como:
\[
S = A \times A = \{(a, b) \mid a, b \in A\}
\]
En esta relación, cualquier par de elementos \( a \) y \( b \) están relacionados, por lo que solo hay una clase de equivalencia que contiene a todos los elementos de \( A \).

\end{solution}