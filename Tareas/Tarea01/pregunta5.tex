\begin{solution}
\textbf{Demostración:}

Sea $\overline{X}$ el conjunto de todas las consecuencias lógicas de una colección $X$ de fórmulas, es decir, 
\[
\overline{X} = \{ \varphi \mid \Gamma \vdash \varphi \}.
\]
Queremos demostrar que 
\[
\overline{\Gamma \cap \Delta} \subseteq \overline{\Gamma} \cap \overline{\Delta}.
\]

Supongamos que $\varphi \in \overline{\Gamma \cap \Delta}$. Esto significa que existe una deducción de $\varphi$ a partir de las fórmulas en $\Gamma \cap \Delta$ utilizando las reglas de inferencia de $\mathscr{R}$. Es decir,
\[
\Gamma \cap \Delta \vdash \varphi.
\]
Dado que $\Gamma \cap \Delta \subseteq \Gamma$, esto implica que también podemos deducir $\varphi$ a partir de las fórmulas en $\Gamma$, es decir,
\[
\Gamma \vdash \varphi,
\]
lo que significa que $\varphi \in \overline{\Gamma}$.

De manera similar, dado que $\Gamma \cap \Delta \subseteq \Delta$, también podemos deducir $\varphi$ a partir de las fórmulas en $\Delta$, es decir,
\[
\Delta \vdash \varphi,
\]
lo que significa que $\varphi \in \overline{\Delta}$.

Por lo tanto, tenemos que $\varphi \in \overline{\Gamma}$ y $\varphi \in \overline{\Delta}$, lo que implica que 
\[
\varphi \in \overline{\Gamma} \cap \overline{\Delta}.
\]
En consecuencia, hemos demostrado que 
\[
\overline{\Gamma \cap \Delta} \subseteq \overline{\Gamma} \cap \overline{\Delta}.
\]
\qed

\end{solution}