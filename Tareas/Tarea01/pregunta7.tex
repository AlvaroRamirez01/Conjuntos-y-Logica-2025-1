\begin{solution}
P.D: Probar que $\psi, \neg \psi \vdash \varphi$ para cualquier fórmula $\varphi$.

Suponemos que $\psi$ es verdadera.
\[
\psi \quad \text{(hipótesis)}
\]

Suponemos también que $\neg \psi$ es verdadera.
\[
\neg \psi \quad \text{(hipótesis)}
\]

Por definición de negación, $\neg \psi$ implica que $\psi \to \bot$ (donde $\bot$ denota contradicción o falsedad).
\[
\neg \psi \equiv \psi \to \bot
\]

Usamos Modus Ponens (MP) con $\psi$ y $\psi \to \bot$, lo que nos permite derivar $\bot$.
\[
\bot \quad \text{(MP aplicado a $\psi$ y $\psi \to \bot$)}
\]

Ahora, aplicamos la regla conocida como **explosión** (también llamada ex falso quodlibet), que nos dice que de una contradicción ($\bot$), podemos derivar cualquier fórmula, en este caso, $\varphi$.
\[
\varphi \quad \text{(ex falso quodlibet)}
\]

Concluimos que: De las hipótesis $\psi$ y $\neg \psi$, hemos derivado $\varphi$. Por lo tanto, se tiene que:
\[
\psi, \neg \psi \vdash \varphi
\]

Esto completa la demostración.
\end{solution}