\begin{solution}
Para resolver este problema, podemos hacer la siguiente pregunta al residente local:

\textbf{Pregunta:} "Si te preguntara cuál es el camino correcto a la capital, ¿me dirías que es el de la derecha?"

Esta pregunta es válida independientemente de si el residente siempre dice la verdad o siempre miente, porque:

\begin{itemize}
    \item Si el residente dice la verdad, responderá 'Sí' si el camino correcto es el de la derecha, y 'No' si es el de la izquierda.
    \item Si el residente miente, habría mentido sobre el camino correcto si lo hubieras preguntado directamente. Pero como le estás preguntando lo que él respondería en ese caso, te dará la respuesta opuesta a su mentira, lo que también te llevará al camino correcto.
\end{itemize}

En ambos casos, la respuesta que recibas será la verdad sobre cuál es el camino correcto. Si te responde 'Sí', entonces el camino correcto es el de la derecha; si te responde 'No', entonces el camino correcto es el de la izquierda.

\end{solution}