\begin{solution}
% \begin{enumerate}
%     \item \( \alpha \to (\alpha \to \alpha) \) \\
%     \textbf{Prueba:} \\
%     Usamos el \textbf{Axioma 1}: \( A \to (B \to A) \). \\
%     Sustituimos \( A \) por \( \alpha \) y \( B \) por \( \alpha \), obteniendo:
%     \[
%     \alpha \to (\alpha \to \alpha)
%     \]
%     Por lo tanto, esta fórmula \textbf{es un teorema} del cálculo proposicional.

%     \item \( (\alpha \to \alpha) \to \alpha \) \\
%     \textbf{Prueba:} \\
%     No parece posible derivar \( (\alpha \to \alpha) \to \alpha \) directamente a partir de los axiomas proporcionados. Esta fórmula no es válida en general, ya que suponer que \( \alpha \to \alpha \) implica \( \alpha \) no es siempre cierto. \\
%     Por lo tanto, esta fórmula \textbf{no es un teorema} del cálculo proposicional.

%     \item \( (\neg \alpha \to \beta) \to (\neg \beta \to \neg \alpha) \) \\
%     \textbf{Prueba:} \\
%     Utilizamos el \textbf{Axioma 3}: \( (\neg B \to \neg A) \to ((\neg B \to A) \to A) \). \\
%     Hacemos la sustitución de variables: \( B = \beta \) y \( A = \alpha \), obteniendo:
%     \[
%     (\neg \beta \to \neg \alpha) \to ((\neg \beta \to \alpha) \to \alpha)
%     \]
%     Esta fórmula muestra una estructura relacionada con la contraposición. Además, el teorema \( (\neg \alpha \to \beta) \to (\neg \beta \to \neg \alpha) \) es un teorema clásico conocido como \textit{contraposición}. \\
%     Por lo tanto, esta fórmula \textbf{sí es un teorema} del cálculo proposicional.

%     \item \( (\alpha \to \gamma) \to (\neg \alpha \to \beta) \) \\
%     \textbf{Prueba:} \\
%     No parece posible derivar esta fórmula directamente utilizando los tres axiomas dados. No hay una conexión clara entre \( \alpha \to \gamma \) y \( \neg \alpha \to \beta \). \\
%     Por lo tanto, esta fórmula \textbf{no es un teorema} del cálculo proposicional.
% \end{enumerate}
\end{solution}