\begin{solution}
    \begin{itemize}
        \item $(A \leftrightarrow (B \leftrightarrow C))$, $((A \land (B \land C)) \lor (\neg A \land (\neg B \land \neg C)))$. 
        Fórmulas:
        \begin{itemize}
            \item $F_1: (A \leftrightarrow (B \leftrightarrow C))$
            \item $F_2: ((A \land (B \land C)) \lor (\neg A \land (\neg B \land \neg C)))$
        \end{itemize}   

        Vamos a valuar $F_1$ y $F_2$ con los siguientes valores:
        \begin{itemize}
            \item $A = \text{verdadero}$.
            \item $B = \text{verdadero}$. 
            \item $C = \text{verdadero}$.
        \end{itemize}

        Para $F_1$:
        \[
        A = \text{verdadero}, B = \text{verdadero}, C = \text{verdadero}
        \]
        Evaluamos \( B \leftrightarrow C \):
        \[
        B \leftrightarrow C = \text{verdadero}
        \]
        Luego, \( A \leftrightarrow (B \leftrightarrow C) \) es:
        \[
        \text{verdadero} \leftrightarrow \text{verdadero} = \text{verdadero}
        \]
        Entonces, \( F_1 \) es \textbf{verdadera} en esta valuación.
    
        Para $F_2$:
        \[
        A = \text{verdadero}, B = \text{verdadero}, C = \text{verdadero}
        \]
        El primer término \( A \land (B \land C) \) es:
        \[
        A \land (B \land C) = \text{verdadero} \land (\text{verdadero} \land \text{verdadero}) = \text{verdadero}
        \]
        Por lo tanto, \( F_2 \) es \textbf{verdadera}.

        Se anexa la tabla de verdad para comprobar si existe otros casos donde $F_1$ y $F_2$ se hacen verdaderas:

        \begin{center}
            \begin{tabular}{|c|c|c|c|c|}
            \hline
            $A$ & $B$ & $C$ & $F_1:(A \leftrightarrow (B \leftrightarrow C) $ & $F_2: ((A \land (B \land C)) \lor (\neg A \land (\neg B \land \neg C)))$ \\ \hline
            \text{V} & \text{V} & \text{V} & \text{V} & \text{V} \\ \hline
            \text{V} & \text{V} & \text{F} & \text{F} & \text{F} \\ \hline
            \text{V} & \text{F} & \text{V} & \text{F} & \text{F} \\ \hline
            \text{V} & \text{F} & \text{F} & \text{V} & \text{F} \\ \hline
            \text{F} & \text{V} & \text{V} & \text{F} & \text{F} \\ \hline
            \text{F} & \text{V} & \text{F} & \text{V} & \text{F} \\ \hline
            \text{F} & \text{F} & \text{V} & \text{F} & \text{F} \\ \hline
            \text{F} & \text{F} & \text{F} & \text{V} & \text{V} \\ \hline
            \end{tabular}
        \end{center}

    
        Conclusión:
            En esta valuación, \textbf{ambas} fórmulas \( F_1 \) y \( F_2 \) son \textbf{verdaderas}, por lo tanto esta valuación no demuestra que una implique a la otra.
    
        \item Determina si las fórmulas $(A \to (B \leftrightarrow C))$ y $((A \to B) \leftrightarrow (A \to C))$ son lógicamente equivalentes.

        \textbf{Objetivo:} Determinar si las fórmulas 
\[
F_1: (A \to (B \leftrightarrow C)) \quad \text{y} \quad F_2: ((A \to B) \leftrightarrow (A \to C))
\]
son lógicamente equivalentes.

Para hacerlo, realizaremos una tabla de verdad considerando todas las posibles valuaciones de las variables $A$, $B$, y $C$.

\begin{center}
\begin{tabular}{|c|c|c|c|c|}
\hline
$A$ & $B$ & $C$ & $F_1: (A \to (B \leftrightarrow C))$ & $F_2: ((A \to B) \leftrightarrow (A \to C))$ \\ \hline
\text{V} & \text{V} & \text{V} & \text{V} & \text{V} \\ \hline
\text{V} & \text{V} & \text{F} & \text{F} & \text{F} \\ \hline
\text{V} & \text{F} & \text{V} & \text{F} & \text{F} \\ \hline
\text{V} & \text{F} & \text{F} & \text{V} & \text{V} \\ \hline
\text{F} & \text{V} & \text{V} & \text{V} & \text{V} \\ \hline
\text{F} & \text{V} & \text{F} & \text{V} & \text{V} \\ \hline
\text{F} & \text{F} & \text{V} & \text{V} & \text{V} \\ \hline
\text{F} & \text{F} & \text{F} & \text{V} & \text{V} \\ \hline
\end{tabular}
\end{center}

\textbf{Conclusión:} Como podemos observar en la tabla de verdad, los valores de $F_1$ y $F_2$ coinciden para todas las posibles valuaciones de $A$, $B$, y $C$. Por lo tanto, las fórmulas $F_1: (A \to (B \leftrightarrow C))$ y $F_2: ((A \to B) \leftrightarrow (A \to C))$ son \textbf{lógicamente equivalentes}.

    \end{itemize}
\end{solution}