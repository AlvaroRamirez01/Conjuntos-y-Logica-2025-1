\begin{solution}
Dada la interpretación \( A = \{1, 2, 3, 4\} \) con las siguientes asignaciones:
\begin{itemize}
    \item $R \mapsto \{(1, 2), (1, 3), (2, 3), (3, 3), (3, 4), (4, 1)\}$
    \item $S \mapsto \{(1, 1), (1, 2), (1, 3), (1, 4), (3, 4)\}$
    \item $c \mapsto 1$
    \item $d \mapsto 2$
\end{itemize}

Analizamos cada fórmula para determinar si es verdadera en $A$:

1. \(\forall x_1 \forall x_2 \forall x_3 ((S(x_1, x_2) \land S(x_2, x_3)) \to R(x_1, x_3))\)

\begin{itemize}
    \item Esta fórmula verifica si, para cualquier \( x_1, x_2, x_3 \), si \( S(x_1, x_2) \) y \( S(x_2, x_3) \) son verdaderas, entonces \( R(x_1, x_3) \) debe ser verdadera.
    \item En $A$, esto no es verdadero en general. Por ejemplo, \( S(1, 1) \land S(1, 2) \), pero no es cierto que \( R(1, 2) \), por lo que la fórmula es \textbf{falsa}.
\end{itemize}

2. \(\forall x_1 \forall x_2 (R(x_1, x_2) \to \neg R(x_2, x_1))\)

\begin{itemize}
    \item Esta fórmula verifica que si \( R(x_1, x_2) \) es verdadera, entonces \( R(x_2, x_1) \) debe ser falsa.
    \item En $A$, la fórmula es verdadera, ya que para cada par \( (x_1, x_2) \), si \( R(x_1, x_2) \), entonces no es cierto que \( R(x_2, x_1) \), como puede verificarse en las asignaciones de $R$. Por lo tanto, esta fórmula es \textbf{verdadera}.
\end{itemize}

3. \(\forall x_1 \forall x_2 (\neg S(x_1, x_2) \to \neg R(x_2, x_1))\)

\begin{itemize}
    \item Verifica que si \( S(x_1, x_2) \) es falsa, entonces \( R(x_2, x_1) \) también debe ser falsa.
    \item En $A$, esta fórmula es \textbf{verdadera}, ya que para cada par \( (x_1, x_2) \), si \( S(x_1, x_2) \) no es verdadero, tampoco lo es \( R(x_2, x_1) \).
\end{itemize}

4. \(\forall x_1 \forall x_2 (\exists x_3 (R(x_1, x_3) \land R(x_3, x_2) \to R(x_1, x_2)))\)

\begin{itemize}
    \item Verifica que para cualquier \( x_1 \) y \( x_2 \), existe un \( x_3 \) tal que si \( R(x_1, x_3) \land R(x_3, x_2) \), entonces \( R(x_1, x_2) \).
    \item En $A$, esta fórmula es \textbf{falsa}, ya que no siempre se cumple. Por ejemplo, \( R(1, 2) \land R(2, 3) \), pero no se sigue que \( R(1, 3) \), ya que no existe tal \( x_3 \) que lo cumpla en todos los casos.
\end{itemize}

5. \(\forall x_1 \forall x_2 (\exists x_3 (R(x_1, x_3) \land S(x_3, x_2) \to R(x_1, x_2)))\)

\begin{itemize}
    \item Verifica que si existe un \( x_3 \) tal que \( R(x_1, x_3) \land S(x_3, x_2) \), entonces \( R(x_1, x_2) \).
    \item Esta fórmula es \textbf{falsa}, ya que no siempre es cierto que \( R(x_1, x_2) \) sigue de la combinación de relaciones entre \( R \) y \( S \). Por ejemplo, \( R(1, 2) \land S(2, 2) \), pero no \( R(1, 2) \).
\end{itemize}

6. \(\forall x_1 \forall x_2 (\exists x_3 (S(x_1, x_3) \land R(x_3, x_2) \to R(x_1, x_2)))\)

\begin{itemize}
    \item Verifica que si existe un \( x_3 \) tal que \( S(x_1, x_3) \land R(x_3, x_2) \), entonces \( R(x_1, x_2) \).
    \item Esta fórmula es \textbf{falsa}, ya que no siempre existe tal \( x_3 \) que conecte \( S \) y \( R \) de esta manera. Por ejemplo, \( S(1, 1) \land R(1, 2) \), pero no se sigue que \( R(1, 2) \).
\end{itemize}

7. \(\forall x_1 \forall x_2 (\exists x_3 (R(x_1, x_3) \land R(x_3, x_2) \to S(x_1, x_2)))\)

\begin{itemize}
    \item Verifica que si existe un \( x_3 \) tal que \( R(x_1, x_3) \land R(x_3, x_2) \), entonces \( S(x_1, x_2) \).
    \item Esta fórmula es \textbf{falsa}, ya que no siempre \( S(x_1, x_2) \) se sigue de una cadena de relaciones \( R \). Por ejemplo, \( R(1, 2) \land R(2, 3) \), pero \( S(1, 3) \) no es verdadero.
\end{itemize}

8. \(\forall x_1 ((x_1 = c) \to \exists x_2 R(x_2, x_1))\)

\begin{itemize}
    \item Verifica que si \( x_1 = c \), entonces existe algún \( x_2 \) tal que \( R(x_2, x_1) \).
    \item Esta fórmula es \textbf{verdadera}, ya que \( c = 1 \) y existe \( x_2 = 4 \) tal que \( R(4, 1) \).
\end{itemize}

\textbf{Conclusión:}
Las fórmulas verdaderas en $A$ son:
\begin{itemize}
    \item $\forall x_1 \forall x_2 (R(x_1, x_2) \to \neg R(x_2, x_1))$
    \item $\forall x_1 \forall x_2 (\neg S(x_1, x_2) \to \neg R(x_2, x_1))$
    \item $\forall x_1 ((x_1 = c) \to \exists x_2 R(x_2, x_1))$
\end{itemize}

\end{solution}