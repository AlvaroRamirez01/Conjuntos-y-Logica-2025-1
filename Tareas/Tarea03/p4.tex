\begin{solution}
    \begin{enumerate}
      \item ¿Existe \( g : \mathbb{N} \rightarrow \mathbb{N} \) tal que \( g \neq \text{Id}_{\mathbb{N}} \) y \( g \circ g = g \)?
      
      Respuesta: Sí, existe tal función.
      
      \textbf{Explicación:}
      
      Una función que satisface \( g \circ g = g \) se denomina idempotente. Queremos encontrar una función idempotente que no sea la identidad.
      
      \textbf{Ejemplo de función:}
      
      Definamos \( g : \mathbb{N} \rightarrow \mathbb{N} \) como:
      
      \[
      g(n) = \begin{cases}
      0, & \text{si } n \text{ es impar} \\
      n, & \text{si } n \text{ es par}
      \end{cases}
      \]
      
      \textbf{Verificación:}
      
      \( g \neq \text{Id}_{\mathbb{N}} \): Porque \( g(n) \) cambia los números impares a 0, por lo que no es la función identidad.
        
      Idempotencia (\( g \circ g = g \)):
      
      Para todo \( n \in \mathbb{N} \):
      
      Si \( n \) es par:
          \[
          g(g(n)) = g(n) = n
          \]

      Si \( n \) es impar:
          \[
          g(g(n)) = g(0) = g(0) = 0 = g(n)
          \]
      
      En ambos casos, \( g(g(n)) = g(n) \), por lo que \( g \circ g = g \).
      
      \textbf{Conclusión:} Existe al menos una función \( g \) que cumple las condiciones dadas.
      
      \item ¿Existe \( g : \mathbb{N} \rightarrow \mathbb{N} \) biyectiva tal que \( g \neq \text{Id}_{\mathbb{N}} \) y \( g \circ g = g \)?
      
      \textbf{Respuesta:} No, no existe tal función.
      
      \textbf{Explicación:}
      
      Supongamos que existe una función biyectiva \( g : \mathbb{N} \rightarrow \mathbb{N} \) tal que \( g \circ g = g \) y \( g \neq \text{Id}_{\mathbb{N}} \).
      
      \textbf{Demostración por contradicción:}
      
      \( g \) es biyectiva, por lo que tiene una función inversa \( g^{-1} \).
      
      Dado que \( g \circ g = g \), podemos aplicar \( g^{-1} \) a ambos lados:
      
        \[
        g^{-1} \circ g \circ g = g^{-1} \circ g
        \]
      
      Simplificando:
      
        \[
        g \circ g^{-1} \circ g = \text{Id}_{\mathbb{N}} \circ g = g
        \]
      
      Pero dado que \( g^{-1} \circ g = \text{Id}_{\mathbb{N}} \), tenemos:
      
        \[
        \text{Id}_{\mathbb{N}} \circ g = g
        \]
      
      Esto implica que \( g = g \), lo cual es siempre cierto. Sin embargo, no hemos llegado a una contradicción aún.
      
      Consideremos que \( g \) es idempotente y biyectiva. La única función biyectiva idempotente es la identidad. Esto se debe a que si \( g \) es idempotente (\( g \circ g = g \)) y biyectiva, entonces para todo \( n \in \mathbb{N} \):
      
        \[
        g(n) = g(g(n))
        \]
      
      Como \( g \) es inyectiva, esto implica que:
      
        \[
        n = g(n)
        \]
      
      Por lo tanto, \( g \) es la identidad, lo cual contradice \( g \neq \text{Id}_{\mathbb{N}} \).
      
      Conclusión: No existe una función biyectiva \( g \) distinta de la identidad que sea idempotente.
      
      \item ¿Existe \( g : \mathbb{N} \rightarrow \mathbb{N} \) biyectiva tal que \( g \neq \text{Id}_{\mathbb{N}} \) y \( g \circ g = \text{Id}_{\mathbb{N}} \)?
      
      \textbf{Respuesta:} Sí, existe tal función.
      
      \textbf{Explicación:}
      
      Una función que satisface \( g \circ g = \text{Id}_{\mathbb{N}} \) se denomina involutiva. Queremos encontrar una función biyectiva involutiva que no sea la identidad.
      
      \textbf{Ejemplo de función:}
      
      Definamos \( g : \mathbb{N} \rightarrow \mathbb{N} \) como sigue:
      
      \[
      g(n) = \begin{cases}
      n + 1, & \text{si } n \text{ es par} \\
      n - 1, & \text{si } n \text{ es impar}
      \end{cases}
      \]
      
      \textbf{Verificación:}
      
      \( g \) es biyectiva:
      
      \textbf{Inyectividad:} Supongamos que \( g(n_1) = g(n_2) \). Entonces:
      
      Si ambos \( n_1 \) y \( n_2 \) son pares:
      \[n_1 + 1 = n_2 + 1 \implies n_1 = n_2\]
      
      Si ambos son impares: 
      \[n_1 - 1 = n_2 - 1 \implies n_1 = n_2\]
      
      Si uno es par y otro impar, sus imágenes serán distintas, pues uno será \( n + 1 \) y otro \( n - 1 \).
      
      \textbf{Sobreyectividad:} Para cualquier \( m \in \mathbb{N} \):
      
      Si \( m \) es par, entonces \( m = g(m - 1) \) (porque \( m - 1 \) es impar).
      
      Si \( m \) es impar, entonces \( m = g(m + 1) \) (porque \( m + 1 \) es par).
      
      \( g \neq \text{Id}_{\mathbb{N}} \):
      
      Porque, por ejemplo, \( g(0) = 1 \neq 0 \).
      
      Involutividad (\( g \circ g = \text{Id}_{\mathbb{N}} \)):
      
      Para todo \( n \in \mathbb{N} \):
      
      \begin{itemize}
        \item Si \( n \) es par:
        
        \( g(n) = n + 1 \) (impar).
            
        \( g(g(n)) = g(n + 1) \).
        
        Como \( n + 1 \) es impar, entonces:
        
        \[g(n + 1) = (n + 1) - 1 = n\]
        
        \item Si \( n \) es impar:
        
        \( g(n) = n - 1 \) (par).
        
        \( g(g(n)) = g(n - 1) \).
        
        Como \( n - 1 \) es par, entonces:
        \[g(n - 1) = (n - 1) + 1 = n\]
      \end{itemize}
      
        En ambos casos, \( g(g(n)) = n \), por lo que \( g \circ g = \text{Id}_{\mathbb{N}} \).
      
      \textbf{Conclusión:} Existe una función biyectiva \( g \) distinta de la identidad que es involutiva.
    \end{enumerate}
\end{solution}